\documentclass[
	fontsize=10pt,
	twoside=true,
	open=any,
	%chapterprefix=true,
	%chapterentrydots=true,
	numbers=noenddot,
	%draft=true,
	%overfullrule=true,
]{kaobook}

% Set the language
\usepackage[english]{babel} % Load characters and hyphenation
\usepackage[english=british]{csquotes} % English quotes
\usepackage[utf8]{inputenc}

% Load packages for testing
\usepackage{blindtext}
%\usepackage{showframe} % Uncomment to show boxes around the text area, margin, header and footer
%\usepackage{showlabels} % Uncomment to output the content of \label commands to the document where they are used

% Load the bibliography package
\usepackage{styles/kaobiblio}
%\addbibresource{main.bib} % Bibliography file

% Load mathematical packages for theorems and related environments. NOTE: choose only one between 'mdftheorems' and 'plaintheorems'.
\usepackage{styles/mdftheorems}
%\usepackage{styles/plaintheorems}

\usepackage{color}

\usepackage{listings}
\lstset{ % General setup for the package
    language=Perl,
    basicstyle=\small\sffamily,
    numbers=left,
    numberstyle=\tiny,
    frame=tb,
    tabsize=4,
    columns=fixed,
    showstringspaces=false,
    showtabs=false,
    keepspaces,
    commentstyle=\color{red},
    keywordstyle=\color{blue}
}
\lstdefinestyle{numbers}{numbers=left, stepnumber=1, numberstyle=\tiny, numbersep=10pt}
\lstdefinestyle{nonumbers}{numbers=none}

\graphicspath{{examples/documentation/images/}{images/}} % Paths in which to look for images

\makeindex[columns=3, title=Alphabetical Index, intoc] % Make LaTeX produce the files required to compile the index

\makeglossaries % Make LaTeX produce the files required to compile the glossary

\makenomenclature % Make LaTeX produce the files required to compile the nomenclature

\renewcommand{\eg}{\emph{e.g.,}}

\newcommand{\ames}{\texttt{\%ames}}
\newcommand{\behn}{\texttt{\%behn}}
\newcommand{\clay}{\texttt{\%clay}}
\newcommand{\dill}{\texttt{\%dill}}
\newcommand{\eyre}{\texttt{\%eyre}}
\newcommand{\ford}{\texttt{++ford}}
\newcommand{\gall}{\texttt{\%gall}}
\newcommand{\iris}{\texttt{\%iris}}
\newcommand{\jael}{\texttt{\%jael}}
\newcommand{\lull}{\texttt{\%lull}}
\newcommand{\zuse}{\texttt{\%zuse}}

\newcommand{\ask}{\texttt{\%ask}}
\newcommand{\graphstore}{\texttt{\%graph-store}}
\newcommand{\mock}{\texttt{++mock}}
\newcommand{\say}{\texttt{\%say}}
\newcommand{\unix}{\texttt{\%unix}}

\newcommand{\nullchr}{\texttt{\textasciitilde}}
\newcommand{\patc}{\texttt{@c}}
\newcommand{\patp}{\texttt{@p}}
\newcommand{\patt}{\texttt{@t}}
\newcommand{\patub}{\texttt{@ub}}
\newcommand{\patud}{\texttt{@ud}}
\newcommand{\patux}{\texttt{@ux}}

\newcommand{\ub}{${0b}$}
\newcommand{\ux}{${0x}$}

\newcommand{\yes}{\texttt{\%.y}}
\newcommand{\no}{\texttt{\%.n}}

\newcommand{\irrtis}{\texttt{=(a b)}}

\newcommand{\gold}{\texttt{\%gold}}
\newcommand{\iron}{\texttt{\%iron}}
\newcommand{\lead}{\texttt{\%lead}}
\newcommand{\zinc}{\texttt{\%zinc}}

\newcommand{\wut}{\texttt{?}}

\newcommand{\pcenhep}{\texttt{\%-} "cenhep"}
\newcommand{\pdotket}{\texttt{.\^} "dotket"}
\newcommand{\pdottar}{\texttt{.*} "dottar"}
\newcommand{\psigpam}{\texttt{~\&} "sigpam"}
\newcommand{\pwuttis}{\texttt{?=} "wuttis"}
\newcommand{\pzapzap}{\texttt{!!} "zapzap"}
\newcommand{\pzaptis}{\texttt{!=} "zaptis"}
\newcommand{\pzapcom}{\texttt{!,} "zapcom"}

\newcommand{\cenhep}{\texttt{\%-}}
\newcommand{\dottar}{\texttt{.*}}
\newcommand{\dottis}{\texttt{.=}}
\newcommand{\sigpam}{\texttt{~\&}}
\newcommand{\wuttis}{\texttt{?=}}
\newcommand{\zapzap}{\texttt{!!}}

% Reset sidenote counter at chapters
%\counterwithin*{sidenote}{chapter}

%----------------------------------------------------------------------------------------

\begin{document}

%----------------------------------------------------------------------------------------
%	BOOK INFORMATION
%----------------------------------------------------------------------------------------

\titlehead{}
\subject{}

\title{An Approach to Developing on Urbit}
\subtitle{}

\author[N E Davis]{N E Davis} %\thanks{University of Illinois}}

\date{\today}

\publishers{Malancandra \& Sons}

%----------------------------------------------------------------------------------------

\frontmatter % Denotes the start of the pre-document content, uses roman numerals

%----------------------------------------------------------------------------------------
%	OPENING PAGE
%----------------------------------------------------------------------------------------

%\makeatletter
%\extratitle{
%	% In the title page, the title is vspaced by 9.5\baselineskip
%	\vspace*{9\baselineskip}
%	\vspace*{\parskip}
%	\begin{center}
%		% In the title page, \huge is set after the komafont for title
%		\usekomafont{title}\huge\@title
%	\end{center}
%}
%\makeatother

%----------------------------------------------------------------------------------------
%	COPYRIGHT PAGE
%----------------------------------------------------------------------------------------

\makeatletter
\uppertitleback{\@titlehead} % Header

\lowertitleback{
	\textbf{Copyright ©2021 by N E Davis}\\

	\medskip

	\textbf{Colophon} \\
	This document was typeset with the help of \href{https://sourceforge.net/projects/koma-script/}{\KOMAScript} and \href{https://www.latex-project.org/}{\LaTeX} using the \href{https://github.com/fmarotta/kaobook/}{kaobook} class.

	The source code of this book is available at:\\\url{https://github.com/davis68/urbit-textbook}

	\medskip

	\textbf{Publisher} \\
	First printed in July 2022 by \@publishers
}
\makeatother

%----------------------------------------------------------------------------------------
%	DEDICATION
%----------------------------------------------------------------------------------------

\dedication{
	Lights All Askew in the Heavens. \\
	Stars Not Where They Seemed or Were Calculated to Be. \\
	A BOOK FOR 12 WISE MEN. \\
	No More in All the World Could Comprehend It. \\
	\flushright -- \href{https://en.wikisource.org/wiki/The_New_York_Times/Lights_All_Askew_in_the_Heavens}{\emph{The New York Times}, November 19, 1919}
}

%----------------------------------------------------------------------------------------
%	OUTPUT TITLE PAGE AND PREVIOUS
%----------------------------------------------------------------------------------------

% Note that \maketitle outputs the pages before here

% If twoside=false, \uppertitleback and \lowertitleback are not printed
% To overcome this issue, we set twoside=semi just before printing the title pages, and set it back to false just after the title pages
\KOMAoptions{twoside=semi}
\maketitle
\KOMAoptions{twoside=false}

%----------------------------------------------------------------------------------------
%	PREFACE
%----------------------------------------------------------------------------------------

%\input{chapters/preface.tex}

%----------------------------------------------------------------------------------------
%	TABLE OF CONTENTS & LIST OF FIGURES/TABLES
%----------------------------------------------------------------------------------------

\begingroup % Local scope for the following commands

% Define the style for the TOC, LOF, and LOT
%\setstretch{1} % Uncomment to modify line spacing in the ToC
%\hypersetup{linkcolor=blue} % Uncomment to set the colour of links in the ToC
\setlength{\textheight}{23cm} % Manually adjust the height of the ToC pages

% Turn on compatibility mode for the etoc package
\etocstandarddisplaystyle % "toc display" as if etoc was not loaded
\etocstandardlines % toc lines as if etoc was not loaded

\tableofcontents % Output the table of contents

\listoffigures % Output the list of figures

% Comment both of the following lines to have the LOF and the LOT on different pages
\let\cleardoublepage\bigskip
\let\clearpage\bigskip

\listoftables % Output the list of tables

\endgroup

%----------------------------------------------------------------------------------------
%	MAIN BODY
%----------------------------------------------------------------------------------------

\mainmatter % Denotes the start of the main document content, resets page numbering and uses arabic numbers
\setchapterstyle{kao} % Choose the default chapter heading style

\setchapterpreamble[u]{\margintoc}
\chapter{A Brief Introduction}
\labch{intro}


\section{What We Talk About When We Talk About Urbit}
\labsec{urbittalk}

Urbit is a functional-as-in-language, network-first, compatibility-breaking
operation function (or hosted operating system).  But what does any of this mean?  As we explore Urbit software development throughout this book, keep in mind that every piece of Urbit aims to solve a ambitious battery of critical problems with the existing legacy World Wide Web.

\subsection{A Series of Unfortunate Events}

\paragraph{Centralization}

For most contemporary corporations, whether enterprise-scale or startup, the driving factor for growth and revenue became the number of customers (users) they were able to attract to their platform or app.  Services like \texttt{del.icio.us} (founded 2003) and Flickr (founded 2004) betokened a wave of massive centralization, cemented by Facebook, Google, and Apple in the late aughts.  TODO XXX number of users on each in 2010

As users jostled onto burgeoning social media platforms, their patterns of behavior changed, and more and more social interactions of significance took place within "walled gardens," service platforms that interfaced only poorly with the exterior web.  Vendor lock-in and the nonportability of user data between platforms meant that consumer choice became a byword.  It became (and remains) difficult for any user to find out just what a corporation or even an app knows about them, particularly given the rise of surveilling cookies and data trackers.

The shift to mobile computing starting with the 2007 launch of Apple's iPhone drove a rise in cloud computing and cloud storage.  To many users, the data storage and access permissions on their data became largely illegible.  Sometimes this led to poor assumptions, such as that the custodial corporation would never allow a leak, or that the data would always be backed up safely.  As projects failed (like \texttt{del.icio.us}) or unilaterally changed policies (Tumblr), users permanently lost data.  Given the effort involved in curating tags, bookmarks, images, contacts, and research data, these outcomes frequently amounted in the loss of years of human effort.

\paragraph{Data leaks}

During the 2000s and 2010s, data leaks became so common as to hardly merit notice.  As users flocked to corporate platforms for social media, publishing, photography, dating, and every other aspect of digital life, insufficient attention was given by corporations to both the practical security of user data and the potential fallout of leaks.  Data breaches grew in number ever year, and affected corporations of every size in every industry.

\begin{itemize}
  \item  2013:  Evernote, 50 million records
  \item  2014:  Ebay, 145 million records
  \item  2015:  Ashley Madison, 32 million records
  \item  2016:  Yahoo!, 1 billion records
  \item  2017:  Experian, 147 million records
  \item  2019:  Facebook, 850 million records
  \item  2019:  CapitalOne, 106 million records
\end{itemize}

("Records" does not equal "people" or even "accounts," of course, rendering these numbers mutually incommensurable.  Regardless, the scale staggers the mind.)  Sometimes these breaches were the result of clever social engineering; more frequently, someone forgot to properly salt password hashes or just stored or transmitted them in unencrypted plaintext.  Occasionally, the data were even just left available at a deprecated or forgotten API endpoint.  Identity security is challenging to get right, and those who had custody of user data were frequently subject to moral hazard.
%Data are from [Juliana de Groot, "The History of Data Breaches"](https://digitalguardian.com/blog/history-data-breaches) and [Wikipedia, "List of Data Breaches"](https://en.wikipedia.org/wiki/List_of_data_breaches).

\paragraph{The looming software stack}

A combination of practical manufacturing limits ending Moore's law and a complexifying operating system and software stack led to a long-term stagnation in the perceived speed and fluidity of user experience with computers.  For the most part, even as multicore CPUs become more widespread, software bloat grows more acute with each new operating system version.  For many enterprise developers, there have been insufficient incentives to simplify software rather than to continue making it more complex.  Minimalist software by and large remained the demesne of hackers and code golf enthusiasts.

For instance,
TODO MS Word menu structure and file bloat
% https://winworldpc.com/product/microsoft-office/95

Even websites with visually minimalist aesthetics often presented
\citeauthor{Ceglowski2015}
a
- Optional Reading: \href{https://idlewords.com/talks/website_obesity.htm}{Maciej Cegłowski, "The Website Obesity Crisis"}

- Optional Reading: \href{https://idlewords.com/talks/build_a_better_monster.htm}{Maciej Cegłowski, "Build a Better Monster: Morality, Machine Learning, and Mass Surveillance"}

- Optional Reading: \href{http://marktarver.com/thecathedralandthebizarre.html}{Mark Tarver, "The Cathedral and the Bizarre"}


\paragraph{Security breaches}

As the softwar stack grows, dependencies become opaque to downstream developers and users.  Upstream vulnerabilities have led to zero-day exploits and security breaches.  For instance, in 2014 the popular OpenSSL cryptography package had a bug of two years' standing revealed, Heartbleed.  This flaw in the Transport Layer Security (TLS) exposed memory buffers adjacent to

Instant-messaging protocols relying on the \texttt{libpurple} were impacted by an out-of-bounds write flaw in 2017, potentially permitting denial-of-service attacks or arbitrary code execution.

These two examples are not cherry-picked:  other examples abound.  The point stands that security breaches in the software stack render reliant software vulnerable in unpredictable ways.

\paragraph{Morbidity in open-source software projects}

The rise of the free and open-source software (FOSS) movement has been enormously influential on software development and the end-user experience.  Spearheaded by Richard Stallman's GNU Project and Linus Torvalds' Linux operating system, FOSS rapidly overtook enterprise software offerings in terms of feature parity and upstream utilization.

Unfortunately, open-source software products are frequently broken in ways that are opaque to relatively nontechnical users:

\begin{enumerate}
	\item  FOSS can be construed as operating under a parasitic model.  Most real innovation happens outside of open-source projects, which are often clones of more successful proprietary software packages (LibreOffice/Microsoft Office, GIMP/Adobe Photoshop, Inkscape/Adobe Illustrator), and/or a clever way for a company to farm out development to free community labor (OpenOffice/Oracle, Ubuntu/Canonical, Darwin/Apple).  Thus even FOSS successes are often copies of proprietary antecedents.
  \item  FOSS suffers from [what one observer has dubbed](http://marktarver.com/thecathedralandthebizarre.html) "financial deficiency disease."  Even popular, well-used packages may have little oversight and funding for developers.  As alluded to above, OpenSSL was found in 2014 to have only one full-time developer despite being used by 66\% of Internet users.  Very few companies have succeeded in being FOSS-first (as opposed to FOSS-sometimes).
			%![](https://imgs.xkcd.com/comics/dependency.png)
  \item  FOSS has a hard time responding to customer demands.  The DIY ethos espoused by FOSS developers has often led to demurrage when features are requested.  This is the infamous response, "If you need it, why don't you build it yourself?"  Many users are unable to commit the time to implement the necessary features, and most FOSS projects do not have full-time developers and existing market dynamics sufficient to motivate rapid development.
\end{enumerate}

Even companies that loudly proclaimed support for "data liberation" used this FOSS openness like a lanternfish to later replace an open protocol (\eg~Google Talk) with a proprietary one (Google Hangouts).

Given the cascading stack of legacy software and strange interdependencies, actually getting the secure functionality a user wants often requires a proprietary platform anyway, undermining the aims of FOSS end-user applications and libraries.

\paragraph{Identity is cheap}

Identity itself is cheap:  it costs botnets and spammers nothing to spin up new email addresses and new false identities.  Game-theoretically, spammers thrive in an environment where identity is close to free.

Identity is also dear:  losing a password in a breach can cause at best hours of resetting service logins and at worst the trauma and legal process of recovering from identity theft.

The foregoing summation may read as a bit emotional relative to what the reader is accustomed to reading in an academic textbook.  This is because the structure of our digital life matters as much as the content, and we have been ill-served to date by the incentives and powers that be.

Enter Urbit, stage right.

\subsection{Why Urbit}
\labsec{whyurbit}

\marginnote[2mm]{"Urbit is a clean slate reimagining of the operating system as an 'overlay OS', and a decentralized digital identity system including username, network address and crypto wallet."  (Tlon)}
The Urbit project intends to cut the Gordian knot of user autonomy and privacy.  To this end, the Urbit developers have articulated an approach prioritizing \emph{a legible future-proof program stack}, \emph{data security}, and \emph{cryptographic ownership}.  The ambitious scope of this project—and the evolution of the goals over the decade of the 2010s—has led many to have difficulty grasping what exactly Urbit is all about.  Urbit has been built to provide an Internet where communities can thrive without meddling or interference by third parties, and where what you build truly belongs to you.

\marginnote[2mm]{"[Urbit is] ultimately a hosted OS ([residing] on top of Linux) with an immutable file system with the additional purpose that you build applications distributed-first in a manner where clients store their own data."  ([`scarejunba`](https://news.ycombinator.com/item?id=21672481))}
Your Urbit is a personal server built as a functional-as-in-language operating system that runs as a virtual machine on top of whatever.  (Sometimes the developers refer to this arrangement as a "hosted OS," but they don't mean as in VMWare or VirtualBox or even containerization.)  The Urbit vision is the unification of services and data around a scarce futureproof identity on an innately secure platform.  Briefly put, Urbit requires you to have an \emph{Urbit OS} (which runs your code, stores your data, etc.) and an \emph{Urbit ID} (which secures your ownership of said code and data).

Urbit provides an excellent example of a visionary complex system which is radical (returning to the roots of computing) and forward-looking—and yet still small enough for us to grok all of the major moving parts in the system.  As a "hundred-year computer," Urbit represents how computing could work when computing power approaches negligible cost and bandwidth becomes effectively unlimited (or at least not limiting TODO Ted disagrees), instead focusing on the quality of user experience and user security.  We have found that Urbit is worthy of study in its own right as a compelling clean-state architecture embracing several innovative ideas at its base.

\paragraph{Legible future-proof program stack}

The core of Urbit is an \emph{operating function}, or a functional-as-in-language operating system.  That is, there is a lifecycle function which receives a state and an event, processes the event, and yields a new state.

$$
L
(\sigma,\varepsilon)
\rightarrow
\sigma'
\textrm{.}
$$

The lifecycle function and state are sometimes called the Urbit OS to distinguish them from other aspects of the Urbit project when ambiguity is present.  The Urbit OS lifecycle function is written in a language called Nock and provides operational affordances through the Arvo operating core.  A schematic representation is frequently used:

![](repo:./img/00-urbit-all.png){: width=25%}

At its base, Arvo is an encrypted event log yielding a particular state.  The Nock virtual machine is like Urbit's version of assembler language, and it may in principle be implemented on top of any hardware.  Hoon is Urbit's equivalent of C, a higher-level language with useful macros and APIs for building out software.  Arvo runs atop these definitions.  The Nock VM runs on a binary interpreter layer on top of actual hardware.

The user can think of Urbit OS as a virtual machine which allows everything upstack to be agnostic to the hardware, and handles everything downstack.  (Urbit has sometimes been described as an operating function, and this is what that means.)  Everything is implemented as a unique stateful instance, called a “ship”.

![](repo:./img/00-urbit-exploded.png){: width=50%}

\marginnote[2mm]{We will take a closer look at every part of this system in Chapter~\ref{kernel}.}
The vanes of Arvo provide services:  \ames~provides network interactivity, \clay~provides filesystem services and builds, \jael~provides cryptographic operations, and so forth.  On top of these are built the userspace apps.

![](repo:./img/00-arvo-exploded.png){: width=100%}

As a "hosted OS," Urbit doesn't seek to replace mainline operating systems.  Indeed, presumptively its Nock virtual machine could be run quite close to the bare metal, but Urbit itself would still require some provision of memory management, hardware drivers, and input/output services.  The overarching goal of the Urbit project is instead to replace the insecure messaging and service platforms and protocols used across the current web.

Urbit was designed on the principle that inheriting old platform code is a developer antipattern, given the complexities, vagaries, and vulnerabilities of legacy OSs.  In other words, things must break to be fixed.  Thus Urbit interfaces with other systems, but is a world unto itself internally.

\paragraph{Data security}

TODO

\paragraph{Cryptographic ownership}

We noted above that Urbit OS is an encrypted event log.  Urbit also acts as a universal \href{https://en.wikipedia.org/wiki/Single_sign-on}{single sign-on (SSO)} for the platform and for services instrumented to work with Urbit calls.  Since the Urbit address space is finite, each Urbit ID has inherent value within the system and should be a closely guarded secret.  An instantiation of your Urbit ID is frequently called a \emph{ship}, which lodges on your filesystem at a folder called a \emph{pier}.

\marginnote[2mm]{See Section~\ref{azimuth}~for more details.}
Following in the footsteps of other blockchain technologies, Urbit secures ownership of unique access points in the Urbit address space using Azimuth.  Currently Azimuth is deployed on top of the Ethereum blockchain.

Urbit IDs have mnemonic names attached to them, although fundamentally they are only a number in the address space.  For instance, one example address on Urbit is \texttt{~dopzod-binfyr}, the unique ID one user owns, corresponding to the 32-bit address \texttt{0xeb2a.5a32} in hexadecimal.

On this network-oriented platform, users provide data to service endpoints, retaining their data rather than farming it out.  While no control can be exercised over data once sent out, a proposed reputation system can penalize bad actors in the system with reduced network access and other sanctions.

Let us posit a social operating system, or SOS;  a protocol for network-oriented platforms to utilize to ensure that user requirements are met securely.  If we enumerate user-oriented desiderata for a social operating system, surely the following must rank prominently:

TODO

\marginnote[2mm]{\citeauthor{Thompson1984}}
The system is designed to be transparent.  Something that runs on the Nock VM is of necessity open-source—no binary blobs!  (As with Ken Thompson's "Reflections on Trusting Trust", one can't necessarily trust what's below completely, but that's a problem with any system one did not build oneself from the bare metal up.)

\marginnote[2mm]{At this point, you may feel confused as to what exactly Urbit is.  That's understandable:  it's hard to explain a new system in full until it has started to manifest new and interesting features with broader repercussions.  For comparison, consider the following two interviews from much earlier in the history of the public Internet:

\begin{itemize}
	\item  \href{https://www.youtube.com/watch?v=gipL_CEw-fk}{Bill Gates on David Letterman, 1995} (an attempt to explain the Internet before almost anyone grokked it)
	\item  \href{https://www.youtube.com/watch?v=LaHcOs7mhfU}{David Bowie on the BBC, 1999} (a prophecy which grasps the essence without the technicality)
\end{itemize}

On this basis, it's safe to say that Gates got it, but Bowie ``got it.''  Their interlocutors did not.}
Identity on the current Web is frequently ephemeral and difficult to distinguish from spam.  Identity on Urbit is scarce and stable, much like moving into a house.  The SSO aspect of the system means that you have to remember and use many fewer passwords, and the cryptographic security layers means that as long as you treat your master key like your Bitcoin wallet you will have perpetual security.

The Urbit project does not completely solve all of these problems—for instance, pwned hardware—but it offers a reasonable set of solutions for many of the social and software issues raised by contemporary corporate practice on the World Wide Web.  Many think that it is better to attempt to fix the challenges of data control, privacy, and equity on the current web: \href{https://sovrin.org/}{Sovrin}, \href{https://webassembly.org/}{WebAssembly}, \href{https://ipfs.io/}{InterPlanetary File System}, \href{https://holochain.org/}{Holochain}, \href{https://blog.space.storage/posts/Introducing-Space}{Space}, and \href{https://scuttlebutt.nz/}{Scuttlebutt} each, in their own way, attack the same problems that Urbit seeks to solve, and each is worthy of the reader's further study.

All in all, Urbit like Bitcoin and (the best) blockchain applications seeks to securely deliver on the aims of the old Cypherpunk movement of the 1980s and 1990s:  digital security, digital autonomy.


\section{Azimuth, the Urbit Address Space}
\labsec{azimuth}

Urbit address points are allocated sequentially from $0$ to $2^{128} = 340\,282\,366\,920\,938\,463\,463\,374\,607\,431\,768\,211\,456$ (340 undecillion, $3.4 \times 10^{38}$).  The maximum address space value in this representation is 128 bits wide, although most points in use today are 32 bits wide or smaller.

Urbit is structured with a hierarchy of addressable points, and bands of smaller values have more ``heft'' in the system and broker access for higher-addressed points.  The structure of the address space reveals the governance structure of the Urbit project itself:

\begin{tabular}{llll}
	Bit width & Total number & Title & Role \\ \hline \\
  8-bit points & $256$ & Galaxies & Provide peer discovery and packet routing as well as network protocol governance.  Allocate star address space. \\
	16-bit points & $2^{16}-256 = 65\,280$ & Stars & Routing \& Allocate peer discovery services, handle distribution of software updates, and allocate planet address space. \\
	32-bit points & $2^{32}-2^{16} = 4\,294\,901\,760$ & Planets & Act as primary single-user identities. \\
	64-bit points & $2^{64}-2^{32} \approx 1.84 \times 10^{19}$ & Moons & Act as planet-bound points (devices, bots); each planet has $2^{32}$ moons available to it. \\
	128-bit points & $2^{128}-2^{64} \approx 3.4 \times 10^{38}$ & Comets & Act as anonymous disposable zero-reputation points (bots, single-use accounts); require a star sponsor to access the network, but once online they are persistent. \\
\end{tabular}

\subsection{Naming points}
\labsec{pointnames}

In \citeyear{Zooko2001}, digital cash pioneer Zooko Wilcox-O'Hearn postulated that a namespace cannot simultaneously possess three qualities:

\begin{enumerate}
	\item  distributedness ("in the sense that there is no central authority which can control the namespace, which is the same as saying that the namespace spans trust boundaries"),
	\item  security ("in the sense that name lookups cannot be forced to return incorrect values by an attacker, where the definition of "incorrect" is determined by some universal policy of name ownership"), and
	\item  human legibility (or interpretable by human users).
\end{enumerate}

This trilemma, dubbed Zooko's triangle, laid down a challenge to cryptographic researchers, who spent some effort to empirically refute the postulate.
% https://web.archive.org/web/20011020191610/http://zooko.com/distnames.html

The Urbit ID system resolves Zooko's triangle by using peer-to-peer routing after discovery, by strictly limiting identity as a scarce and reputation-bearing good, and by assigning each addressable point of the 128-bit address space a unique and memor(iz)able name.

Each point receives a unique pronounceable name constructed from a list of 256 prefixes and 256 suffixes.  For instance, point $0$ is \zod, the root sponsoring galaxy of the \ames~network.  In fact, today on Urbit you frequently see the mnemonic address used as the primary pseudonymous identity and username.  The identity problem is thereby solved without restrictive username requirements and collision-avoidance strategies.

Urbit uses a system of mnemonic syllables to uniquely identify each address point.  These mnemonic names, called "`patp`s" after their Hoon representation `@p`, occur in a set of 256 suffixes (such as "zod") and 256 prefixes (such as "lit").  They were selected to be pronounceable but not meaningful.

| Number | Prefix | Suffix |
| ------ | ------ | ------ |
|      0 |    doz |    zod |
|      1 |    mar |    nec |
|      2 |    bin |    bud |
|      3 |    wan |    wes |
|      4 |    sam |    sev |
|    ... |    ... |    ... |
|    254 |    mip |    nev |
|    255 |    fip |    fes |

The first 256 points (the galaxies) simply take their name from the suffix of their address point.  Subsequent points combine values:  for instance, point 256 is ~marzod, point 480 is ~marrem, and point 67,985 is ~fogfel-folden.  (Conventionally, a sigma `~` is used in front of an address.)

\marginnote[2mm]{Sigils are a visual corollary to the mnemonic patp.  Each 32-bit or lower address has a unique sigil, based on the 512 component syllables.  Sigils are not intrinsic to Urbit, but they form part of the metatextual environment that Urbit inhabits and they are frequently used as a means of ready differentiation and identity.  TODO images}

The 256 galaxies have suffix-only names, and all higher addresses have prefix–suffix names.  Two-syllable names always mean the point is a star; four-syllable names are planets.  Comets have rather cumbersome names:  67,985,463,345,234,345 corresponds to ~doztes-nodbel-palleg-ligbep with eight syllables.

The stars which correspond to a galaxy are suffixed with the galaxy's name; planet names are mangled so that one cannot tell which star or galaxy a planet corresponds to at a glance.

If a planet needs to change its sponsor, \href{https://urbit.org/using/operations/using-your-ship/#escaping-a-sponsor}{there is support for changing one's sponsor}, in which another star can assume the role of peer discovery in case a star goes offline (a "dark star").

TODO we in principle care about address when dealing with a strange star or planet for the first time.  A reputation system is under development, but hasn't yet seemed to be necessary.  This is called [`Censures`](https://urbit.org/docs/glossary/censure/).  Plus, at this point, identity is fairly cheap, abundant if not infinite.  (Notably, not so cheap that spammers can thrive.)

Peer discovery, the primary role of stars besides planet allocation, is an important step in responsibly controlling network traffic.  "The basic idea is, you need someone to sponsor your membership on the network. An address that can't find a sponsor is probably a bot or a spammer" ([docs](https://urbit.org/understanding-urbit/)).

A reputation system is available called [`Censures`](https://urbit.org/docs/glossary/censure/).  As spammers and bots are not yet present on the Urbit network in any significant quantity, Censures is not heavily used today.  Galaxies and stars can censure (or lower the reputation of) lower-ranked points as a deterrent to bad behavior (defined as spamming, scamming, and spreading malware).  Since good behavior is the default, only lowering reputation is supported.

- Reading: [Philip Monk `~wicdev-wisryt`, "Designing a Permanent Personal Identity"](https://urbit.org/blog/pki-maze/)


\subsection{Azimuth On-Chain}
\labsec{blockchain}

Azimuth is a public-key infrastructure (PKI) and is currently deployed as a series of smart contracts operating on the Ethereum blockchain.  ``Azimuth is basically two parts, a database of who owns which points, and a set of rules about what points and their owners can do'' (Wolfe-Pauly).  Azimuth points are not interchangeable tokens like ETH or many other cryptocurrencies:  each point has a unique ID, a type, and associated privileges.  The technical blockchain term for this kind of points is a ``non-fungible token'' (NTF).

Azimuth is located on the Ethereum blockchain at address [`0x223c067f8cf28ae173ee5cafea60ca44c335fecb`](https://etherscan.io/address/0x223c067f8cf28ae173ee5cafea60ca44c335fecb) or [`azimuth.eth`](https://etherscan.io/address/azimuth.eth).

Point ownership is secured by the Urbit HD (Hierarchical Deterministic) wallet, a collection of keys and addresses which allows you fine-grained control over accessing and administering your asset.  The Urbit HD wallet is described in more detail in Section~\ref{hd-wallet}.

There is nothing intrinsic about Azimuth which requires Ethereum to work correctly, and in the future Azimuth will probably be moved entirely onto Urbit itself.

- Reading: [Galen Wolfe-Pauly, "Azimuth is On-Chain", through section "Azimuth"](https://urbit.org/blog/azimuth-is-on-chain/)
- Optional Reading: [Ameer Rosic, "What is An Ethereum Token?"](https://blockgeeks.com/guides/ethereum-token/)
- Code: [`azimuth-js`](https://github.com/urbit/azimuth-js)


\subsection{Urbit Ownership and the Crypto Community}

Because Urbit address space is finite, it in principle bears value similar to cryptocurrencies such as Bitcoin.  You should hold your Urbit keys like a dragon's hoard:  once lost, they are irrecoverable.  No one else has a copy of your master ticket, which is the cryptographic information necessary to sell, launch, or administer your planet.

To be clear, we do not herein promote Urbit or ownership of any part thereof as a speculative crypto asset.  Like blockchain and cryptocurrencies, Urbit may carry intrinsic value or it may be only \href{https://en.wikipedia.org/wiki/Wildcat_banking}{so much (digital) paper}.  Right now, you can purchase available Azimuth points on [OpenSea.io](https://opensea.io/assets/urbit-id?query=urbit) and [urbit.live](https://urbit.live/buy), or you can buy directly from someone who has some available.

- Reading: [Wolf Tivy, "Why do Urbit stars cost so much?" (Quora answer)](https://www.quora.com/Why-do-Urbit-stars-cost-so-much)
- Resource: [Urbit Live, "Urbit Network Stats"](https://urbit.live/stats) (set the dates to a much broader range and the currency type to ETH)


\paragraph{Galaxy Ownership and Star Access}

The sale of galaxies formed a role in initially funding Urbit, but to prevent too early sale and to modulate access to the network, most stars are locked by Ethereum smart contracts and unsellable until their unlock date.  Star contracts will unlock through January 2025, at which point the full address space will be available (but not necessarily activated).  Galaxy owners can sell or distribute stars as they see fit, and star owners can parcel out planets.  However, since a star provides peer discovery services, it is imperative that a star with daughter planets remain online and up-to-date.

Much of the Urbit address space is locked and unspawnable to provide an artificial brake on supply and prevent overrunning the available address space.  See "The Value of Urbit Address Space, Part 3" for extensive details on star and planet limitations and the associated Ethereum smart contracts.

![](https://media.urbit.org/site/posts/essays/value-of-address-space-pt3-graph1.png){: width=100%}

If a star ceases to provide peer-to-peer lookup services and software updates, a planet may find itself in a pickle.  "Dark stars" are stars which have spawned daughter planets but are not running anymore.  To mitigate this situation, planets can switch from one sponsoring star and move to another.

- Optional Reading: [Erik Newton `~patnes-rigtyn`, Galen Wolfe-Pauly `~ravmel-ropdyl`, "The Value of Urbit Address Space, Part 1"](https://urbit.org/blog/value-of-address-space-pt1/)
- Optional Reading: [Erik Newton `~patnes-rigtyn`, Galen Wolfe-Pauly `~ravmel-ropdyl`, "The Value of Urbit Address Space, Part 2"](https://urbit.org/blog/value-of-address-space-pt2/)
- Optional Reading: [Erik Newton `~patnes-rigtyn`, Galen Wolfe-Pauly `~ravmel-ropdyl`, "The Value of Urbit Address Space, Part 3"](https://urbit.org/blog/value-of-address-space-pt3/)

The Azimuth PKI is quite sophisticated, and the associated Urbit HD wallet allows for nuance in point management.  The \href{https://bridge.urbit.org/}{Bridge interface} is used to manage point operations in the browser.  We will revisit all three in technical detail in Section~\ref{azimuth:deep}.  We will furthermore examine the internal operations of `azimuth-js` and the Ecliptic contracts.


\section{A Frozen Operating System}
\labsec{frozen}

The philosophy underlying Urbit bears a strange resemblance to mathematics:  rather than running always as fast as one can to stay in the same place (a Red Queen's race), one should instead establish a firm foundation on which to erect all future enterprises.  In this view, the operating system should provide a permanently future-proof platform for launching your applications and storing your data—rather than a pastiche of hardware platforms and network specifications, all of that is hidden, "driver-like."  The OS should explicitly obscure all of that and no reaching beneath the OS should be allowed.

\marginnote[2mm]{Urbit frequently refers to its way of doing things as "Martian."}
From [the docs](https://web.archive.org/web/20140424223249/http://urbit.org/community/articles/martian-computing/):

\begin{quote}
Normally, when normal people release normal software, they count by fractions, and they count up.  Thus, they can keep extending and revising their systems incrementally.  This is generally considered a good thing.  It generally is.

\marginnote[2mm]{Urbit is not, of course, the only system to adopt an asymptotic approach to its final outcome.  \href{http://www.texfaq.org/FAQ-TeXfuture}{Donald Knuth, famous for many reasons but in this particular instance for the typesetting system \TeX, has specified that \TeX~versions incrementally approach $\pi$.}  \TeX~will reach $\pi$ definitively upon the date of Knuth's death, at which point all remaining bugs are instantly transformed into features and the version becomes $\pi$.  The current version of TeX is 3.14159265.}

In some cases, however, specifications needs to be permanently frozen.  This requirement is generally found in the context of standards.  Some standards are extensible or versionable, but some are not.  ASCII, for instance, is perma-frozen.  So is IPv4 (its relationship to IPv6 is little more than nominal—if they were really the same protocol, they'd have the same ethertype).  Moreover, many standards render themselves incompatible in practice through excessive enthusiasm for extensibility.  They may not be perma-frozen, but they probably should be.

The true, Martian way to perma-freeze a system is what I call Kelvin versioning.  In Kelvin versioning, releases count down by integer degrees Kelvin.  At absolute zero, the system can no longer be changed.  At 1K, one more modification is possible.  And so on.  (\cite{Yarvin2017})
\end{quote}

In other words, Urbit is intended to cool towards absolute zero, at which point its specification is locked in forever and no further changes are countenanced.  This doesn't apply to everything in the system—"there simply isn't that much that needs to be versioned with a kelvin" (~nidsut-tomdun)—but it does apply to the most core components in the system.

\marginnote[2mm]{Think of the hypothetical structure of Jupiter:  clouds over a sea of metallic hydrogen over a diamond as big as Earth.
TODO image}

In this light, when we talk about Urbit we talk about three things:

\begin{enumerate}
	\item  Crystalline Urbit (the promised frozen core, 0K)
  \item  Fluid Urbit (the practice, mercurial and turbulent but starting to take shape)
  \item  Mechanical Urbit (the under-the-hood elements, still a chaos lurching into being, although much less primeval than before)
\end{enumerate}


\section{Developing for Urbit}
\labsec{developing}

The primary aim of this textbook is to expound Urbit in sufficient depth that you can approach it as an effective software developer.  We assume previous programming experience of one kind or another, not necessarily in a functional language.

Urbit development can be divided into three cases:

\begin{enumerate}
	\item  Kernel development
	\item  Userspace development, Urbit-side (\gall~and generators)
	\item  Userspace development, client-side (Urbit API)
\end{enumerate}


This guide focuses on getting the reader up to speed on the second development case early, then branches out into the two others.  With a solid foundation in \gall, the reader will be well-equipped to handle demands in either of the other domains.  We encourage the reader to approach each example and exercise in the following spirit:

\begin{enumerate}
  \item  Identify the input and outputs, preferably at the data type level and contents.
	\item  Reason analogically from other Hoon examples available in the text and elsewhere.
	\item  Create and complete an outline of the code content.
	\item  Devise and compose a suitable test suite.
\end{enumerate}

\subsection{Practicalities}
\labsec{in:practicalities}

We recommend organizing all development work discussed in this textbook into a single folder containing code folders, version control repositories of code, data, and so forth.  For convenience, we locate this folder at \texttt{\textasciitilde/urbit}.  Should you choose to install Urbit in this location, you should use the folder \texttt{\textasciitilde/urbit/bin} to contain the Urbit executables.

\paragraph{Development Ships (Fakezods)}

When started directly by the user, Urbit enters a read-evaluate-print loop (REPL) immediately after booting.  This interface, called Dojo, can process many Hoon expressions and provides some support for administering apps, interacting with command-line interface (CLI) agents, and working with the Unix file system and input/output processes.

Urbit ships are commonly divided into \emph{live ships} and \emph{fakezods} (after \zod, the parent galaxy).  Live ships are end-user instances operating on live \ames.  Fakezods are disconnected and keyless ships frequently used in development.  Most of our work in this book should be completed using fakezods, as one can lobotomize one's personal live ship by committing bad agent code to the Urbit file system.

To create a fakezod, one passes the \texttt{urbit} executable the \texttt{-F} ‘fake keys’ flag:

\lstset{language=bash}
\begin{lstlisting}[style=nonumbers]
$ urbit -F zod
\end{lstlisting}

As downloading a boot sequence or \emph{pill} can take some time, we recommend downloading the current pill once, storing it locally, and using it to boot fakezods for a while until it becomes outdated.  The URL for the current pill can be found by starting a fakezod, observing the URL indicated in the boot sequence, and aborting the boot sequence before retrieving the pill using a tool like \texttt{wget} or a web browser.  (Use \CtrlZ~to abort the process.)

\begin{lstlisting}[style=nonumbers]
$ urbit -F tex
~
urbit 1.0
boot: home is /home/davis68/tex
loom: mapped 2048MB
lite: arvo formula 79925cca
lite: core 59f6958
lite: final state 59f6958
boot: downloading pill https://bootstrap.urbit.org/urbit-v1.0.pill

[received keyboard stop signal, exiting]
$ https://bootstrap.urbit.org/urbit-v1.0.pill
\end{lstlisting}

Once the boot sequence has completed on a new fakezod, one can use the \texttt{-B} ‘pill file’ flag to start the fakezod:

\begin{lstlisting}[style=nonumbers]
$ urbit -F zod -B urbit-v1.0.pill
\end{lstlisting}

\marginnote[2mm]{We will discuss the boot sequence in detail in Section~\ref{kr:boot}.}
After a few minutes of boot sequence, the Dojo REPL prompt appears.

\begin{lstlisting}[style=nonumbers]
$ urbit -F zod -B urbit-v1.0.pill
~
urbit 1.0
boot: home is /home/davis68/tex
loom: mapped 2048MB
lite: arvo formula 79925cca
lite: core 59f6958
lite: final state 59f6958
boot: loading pill urbit-v1.0.pill
loom: mapped 2048MB
boot: protected loom
live: logical boot
boot: installed 286 jets
boot: parsing %brass pill
---------------- playback starting ----------------
...
---------------- playback complete ----------------
vere: checking version compatibility
ames: live on 31455 (localhost only)
http: web interface live on http://localhost:8080
http: loopback live on http://localhost:12321
pier (20): live
ames: metamorphosis
~zod:dojo>
\end{lstlisting}

\marginnote[2mm]{The Dojo automatically parses input for validity, so attempting to type some sequences may fail.  This is confusing at first but soon becomes an indispensable validation of newly minted code.}
To verify that the fakezod has loaded correctly, the reader should type some simple primitives:

\begin{lstlisting}[style=nonumbers]
> 1
1
> 'hello mars'
'hello mars'
> "Hello Mars"
"Hello Mars"
\end{lstlisting}

Since an Urbit ship is presumably always-on, shutting down the ship is simply a special case of suspending computation.  Press \CtrlD~to stop the fakezod from running, or \CtrlZ~to force stop.  To start the fakezod again, use the \texttt{urbit} executable with the pier name (the name of the folder created for the fakezod):

\begin{lstlisting}[style=nonumbers]
$ urbit zod
\end{lstlisting}

Since the behavior of a ship is determined by its state, and its state is determined by its boot sequence, a live ship or a fakezod will remain as such perpetually (although one can run a live ship disconnected from the network).  Unless otherwise specified, we assume all development to take place on a fakezod.

\paragraph{Persistent sessions}

When running a ship to which one wishes to connect repeatedly without shutdown (the standard case for usage and an occasional case in development), one should employ a tool such as \texttt{screen} to persist sessions even when a terminal session is not actively connected.

One way to set up this situation is as follows:

\begin{enumerate}
  \item  Download and install \href{TODO}{\texttt{screen}} if it is not already available on your OS platform.
  \item  Start a new screen session with appropriate name:
    \begin{lstlisting}
screen -S sampel-palnet
    \end{lstlisting}
  \item  In this terminal session, start the Urbit ship, whether a fakezod or a live ship.
    \begin{lstlisting}
urbit sampel-palnet
    \end{lstlisting}
  \item  Once the ship is running correctly, once you are ready to detach from the session, press \texttt{Ctrl}+\texttt{A}, then \texttt{d} to disconnect.
  \item  \marginnote[2mm]{Be careful not to start multiple sessions with the same name at the same time or you will need to access your target session via a unique session number as well.}  To connect to the session again, use
    \begin{lstlisting}
screen -r sampel-palnet
    \end{lstlisting}
\end{enumerate}

Sessions do not persist past system shutdown.


\section{Exercises}
\labsec{intro:exercises}

\begin{enumerate}
	\item  Obtain an Urbit ID and set up the Urbit OS.  Use the current installation procedure outlined at \href{https://urbit.org/}{\texttt{urbit.org}}.  One does not need to use a hosting service if one prefers to run Urbit on one's own hardware, but maintaining the live ship in an always-connected state will improve the experience.
	\item  Set up a fakezod for software development.  (We further recommend creating a copy of the pier so that a new fakezod can be started quickly in case of catastrophic failure.)
\end{enumerate}


\pagelayout{wide} % No margins
\addpart{Language Essentials}
\pagelayout{margin} % Restore margins

\setchapterpreamble[u]{\margintoc}
\chapter{Nock, A Combinator Language}
\labch{nock}


\section{Primitive rules and the combinator calculus}
\labsec{primitive}


A combinator calculus is one way of writing primitive computational systems.  Combinatory logic allows one to eliminate the need for variables (unknown quantities like $x$) and thus deal exclusively (?) with pure functions.

one combinator calculus



To understand how Nock expressions produce nouns as pure stateless functions, we need to introduce the \emph{subject}.  The subject is somewhat analogous to a namespace in other programming languages; it encompasses the computational context and the arguments.  Another way to put it is that the subject \emph{is} the argument to the Nock formula:  not all of the subject may be used in evaluating the formula, but it is all present.


Nock is a crash-only language; that is, while it can emit events that are interpretable by the runtime as errors that can be handled, Nock itself fails when an invalid operation occurs.

Nock is a standard of behavior, not necessarily an actual machine.  (It is an actual machine, of course, as a fallback, but the point is that any Nock virtual machine should implement the same behavior.)  We like to think of this analogous to solving a matrix.  Formally, given an equation

$$
A \vec{x} = \vec{b}
$$

the solution should be obtained as

$$
A^{-1} A \vec{x} = A^{-1} \vec{b} \rightarrow \vec{x} = A^{-1} \vec{b}
$$

This is correct, but often computationally inefficient to achieve.  Therefore we use this behavior as a standard definition for $\vec{x}$, but may actually obtain $\vec{x}$ using other more efficient methods.  Keep this in mind with Nock:  one has to know the specification but doesn't have to follow suit to implement it this way (thus, jet-accelerated Nock, Section~\ref{jetting}).

\subsection{Nock 4K}

The current version of Nock, Nock 4K, consists of six primitive rules as well as a handful of compound adjuncts.  The primitive rules are conventionally written in an explanatory pseudocode:

\begin{lstlisting}[style=nonumbers]
*[a 0 b]            /[b a]
*[a 1 b]            b
*[a 2 b c]          *[*[a b] *[a c]]
*[a 3 b]            ?*[a b]
*[a 4 b]            +*[a b]
*[a 5 b c]          =[*[a b] *[a c]]
\end{lstlisting}

with the following operations:

\begin{itemize}
  \item  \texttt{*} is the \emph{evaluate} operator, which operates on a cell of \texttt{[subject formula]};
  \item  \texttt{/} is the \emph{slot} operator or address \texttt{b} of [tree] \texttt{a};
  \item  \texttt{?} is the \emph{cell} operator, testing whether its operand is a cell.
  \item  \texttt{+} is the \emph{increment} operator.
  \item  \texttt{=} is the \emph{equality} operator, checking for structural equality of the operands evaluated against the subject \texttt{a}.
\end{itemize}

It is also instructive to write these as mathematical rules:

\begin{alignat*}{3}
*_{0}&[a](b) &&:= a_{b} \\
*_{1}&[a](b) &&:= b \\
*_{2}&[a](b,c) &&:= *({*[a](b)}, {*[a](c)}) \\
*_{3}&[a](b) &&:= \left\{\begin{matrix} 0 & \text{if cell} \\ 1 & \text{if atom} \end{matrix} \right. \\
*_{4}&[a](b) &&:= {*(a,b) + 1} \\
*_{5}&[a](b,c) &&:= ({*(a,b)} \stackrel{?}{=} {*(a,c)})
\end{alignat*}

where $*$ is the generic evaluate operator.

Each rule is referred to by its number; \eg~"Nock 3" refers to the cell test rule.

Nock operates on unsigned integers, with zero $0$ expressing the null or empty value.  Frequently this is written as a tilde, $\textasciitilde$ or \nullchr.  This value plays a complex role similar to \texttt{NULL} and \texttt{'\textbackslash 0'} in C and other programming languages—although, critically, it is still numeric.

%-------------------------------------------------------------------------------
\subsubsection[Nock 0]{Nock 0, Addressing}
\labsec{nock0}

\begin{lstlisting}[style=nonumbers]
*[a 0 b]            /[b a]
\end{lstlisting}

$$
*_{0}(a,b) := a_{b}
$$

Nock Zero allows the retrieval of nouns against the Nock subject.  Data access requires knowing the address and how to retrieve the corresponding value at that address.  The slot operator expresses this relationship using \texttt{a} as the subject and the atom \texttt{b} as the one-indexed address.

Every structure in Nock is a binary tree.  Elements are enumerated left-to-right starting at $1$ for the entire tree.


One common convention is to store values at the leftward leaves of rightward branches; this produces a cascade of values at addresses $2^{n}-2$.

\marginnote[2mm]{The address of a value in the Nock binary tree has no direct correspondence to its address in physical memory.  This latter is handled by the Nock runtime, avoiding the use of pointers in Nock code.}




By hand,


\marginnote[2mm]{
\dottar~implements Nock Two, which is of course \emph{evaluate}.
}

In the Dojo, you may evaluate this statement using the \pdottar~rune:

\begin{lstlisting}[style=nonumbers]
.*(TODO)
\end{lstlisting}

You may also use \texttt{++mock}

virtualization arm computes a formula.  `++mock` is Nock in Nock, however, so it is not very fast or efficient.

`++mock` returns a tagged cell, which indicates the kinds of things that can go awry:

- `%0` indicates success
- `%1` indicates a blocked calculation
- `%2` indicates a crash with stack trace

`++mock` is used in Gall and Hoon to virtualize Nock calculations and intercept scrys.  It is also used in Aqua, the testnet infrastructure of virtual ships.

%-------------------------------------------------------------------------------
\subsubsection[Nock 1]{Nock 1, Constant Reduction}
\labsec{nock1}

\begin{lstlisting}[style=nonumbers]
*[a 1 b]            b
\end{lstlisting}

$$
*_{1}(a,b) := b
$$

Nock One simply returns the constant value of noun \texttt{b}.

%-------------------------------------------------------------------------------
\subsubsection[Nock 2]{Nock 2, Evaluate}
\labsec{nock2}

\begin{lstlisting}[style=nonumbers]
*[a 2 b c]          *[*[a b] *[a c]]
\end{lstlisting}

$$
*_{2}[a](b,c) := *({*[a](b)}, {*[a](c)})
$$

%-------------------------------------------------------------------------------
\subsubsection[Nock 3]{Nock 3, Test Cell}
\labsec{nock3}

\marginnote[2mm]{
  Although unusual, Nock is by no means the only language to adopt $0$ as the standard of truth.  The POSIX-compliant shells such as Bash adopt the convention that $0$ is \texttt{TRUE}.  So do Ruby and Scheme, although with caveats.
}

Nock Three
zero as true (because there is one way to be right and many ways to be wrong).

%-------------------------------------------------------------------------------
\subsubsection[Nock 4]{Nock 4, Increment}
\labsec{nock4}
%-------------------------------------------------------------------------------
\subsubsection[Nock 5]{Nock 5, Test Equivalence}
\labsec{nock5}

Let us examine some Nock samples by hand and see if we can reconstruct what they do.  We will then create some new short programs and apply them by hand via the Nock rules.


\section{Compound rules}
\labsec{compound}

For the convenience of programmers working directly with Nock (largely the implementers of Hoon), a number of compound rules were defined that reduce to the primitive rules.  These implement slightly higher-order conventions such as a decision operator.  Each of these provide syntactic sugar that render Nock manipulations slightly less cumbersome.

\begin{lstlisting}[style=nonumbers]
*[a 6 b c d]        *[a *[[c d] 0 *[[2 3] 0 *[a 4 4 b]]]]
*[a 7 b c]          *[*[a b] c]
*[a 8 b c]          *[[*[a b] a] c]
*[a 9 b c]          *[*[a c] 2 [0 1] 0 b]
*[a 10 [b c] d]     #[b *[a c] *[a d]]

*[a 11 [b c] d]     *[[*[a c] *[a d]] 0 3]
*[a 11 b c]         *[a c]
\end{lstlisting}

with the following operation:

\begin{itemize}
  \item  \texttt{\#} is the \emph{replace} operator, which edits a noun by replacing part of it with another piece.
\end{itemize}

As mathematical rules, these would be:

\begin{alignat*}{3}
*_{6}&[a](b,c,d) &&:= \left\{ \begin{matrix} *[a](c) & \textrm{if } b \\ *[a](d) & \textrm{otherwise} \end{matrix} \right. \\
*_{7}&[a](b,c) &&:= *[*[a](b)](c) \\
*_{8}&[a](b,c) &&:= *[*[*[a](b)](a)](c) \\  %TODO CHECKME
*_{9}&[a](b,c) &&:= \left\{\begin{matrix} 0 & \text{if cell} \\ 1 & \text{if atom} \end{matrix} \right. \\
*_{10}&[a](b,c,d) &&:= {*(a,b) + 1} \\
*_{11}&[a](b,c,d) &&:= ({*(a,b)} \stackrel{?}{=} {*(a,c)})
*_{11}&[a](b,c) &&:= ({*(a,b)} \stackrel{?}{=} {*(a,c)})
\end{alignat}

where $*$ is the generic evaluate operator.

%-------------------------------------------------------------------------------
\subsubsection[Nock 6]{Nock 6, Conditional Branch}
\labsec{nock6}
%-------------------------------------------------------------------------------
\subsubsection[Nock 7]{Nock 7, Compose}
\labsec{nock7}
%-------------------------------------------------------------------------------
\subsubsection[Nock 8]{Nock 8, Declare Variable}
\labsec{nock8}
%-------------------------------------------------------------------------------
\subsubsection[Nock 9]{Nock 9, Produce Arm of Core}
\labsec{nock9}
%-------------------------------------------------------------------------------
\subsubsection[Nock 10]{Nock 10, Replace}
\labsec{nock10}
%-------------------------------------------------------------------------------
\subsubsection[Nock 11]{Nock 11, Hint to Interpreter}
\labsec{nock11}

\marginnote[2mm]{
There's also a "fake Nock" Rule Twelve, \pdotket, which exposes a namespace into Arvo.  More details on this follow in Section~\ref{TODO}.
}

With Nock under your belt, many of the quirks of Hoon become more legible.  For instance, since everything in Nock is a binary tree, so also everything in Hoon.  Nock also naturally gives rise to cores, which are a way of pairing operations and data in a cell.

Although Nock is the runtime language of Urbit, developers write actual code using Hoon.  Given a Hoon expression, you can produce the equivalent Nock formula using \pzaptis.

After this chapter, you may never write Nock code again.  That's fine!  We need to understand Nock to understand Hoon, but will not need to compose in Nock directly to do any work in Urbit, even low-level work.  (There is no \texttt{inline}~equivalent.)

\begin{lstlisting}[style=nonumbers]
> !=(+(1))
[4 1 1]

> !=((add 1 1))
[8 [9 36 0 1.023] 9 2 10 [6 [7 [0 3] 1 1] 7 [0 3] 1 1] 0 2]
\end{lstlisting}

(Why do these differ so much?  \texttt{++add} is doing a bit more than just adding a raw $1$ to an unsigned integer.  We'll walk through this function later in Section~TODO.)


One last piece is necessary for us to effectively interpret Nock code:  the implicit cons.  Cons is a Lisp function to construct a pair, or what in Nock terms we call a cell.  Many times we find Nock expressions in which the operand is a cell, and so TODO

\subsection{Nock Examples}

We will work through several Nock programs by hand.  Since each Nock program is a pure function and emits no side effects, when we have applied all of the rules to achieve a final value, we are done calculating the expression.

Infamously, Nock does not have a native decrement operator, only an increment (Rule Four).  Let us dissect a simple decrement operation in Nock:

\begin{lstlisting}[style=nonumbers]
> !=(|=(a=@ =+(b=0 |-(?:(=(a +(b)) b $(b +(b)))))))
[ 8
  [1 0]
  [1 8 [1 0] 8 [1 6 [5 [0 30] 4 0 6] [0 6] 9 2 10 [6 4 0 6] 0 1] 9 2 0 1]
  0
  1
]
\end{lstlisting}

which can be restated in one line as

\begin{lstlisting}[style=nonumbers]
[8 [[1 0] [1 8 [1 0] 8 [1 6 [5 [0 30] 4 0 6] [0 6] 9 2 10 [6 4 0 6] 0 1] 9 2 0 1] 0 1]]
\end{lstlisting}

or in many lines as

\begin{lstlisting}[style=numbers]
[8
  [1 0]
  [1 [8
       [1 0]
       [8
         [1 [6
              [5
                [0 30]
                [4 0 6]
              ]
              [0 6]
              [9
                2
                [10
                  [6 4 0 6]
                  [0 1]
                ]
              ]
            ]
           [9 2 0 1]
         ]
       ]
     ]
  ]
  [0 1]
]
\end{lstlisting}

(It's advantageous to see both.)

We can pattern-match a bit to figure out what the pieces of the Nock are supposed to be in higher-level Hoon.  From the Hoon, we can expect to see a few kinds of structures:  a trap, a test, a `sample`.  At a glance, we seem to see Rules One, Five, Six, Eight, and Nine being used.  Let's dig in.

(Do you see all those `0 6` pieces?  Rule Zero means to grab a value from an address, and what's at address `6`?  The `sample`, we'll need that frequently.)

The outermost rule is Rule Eight `*[a 8 b c]→*[[*[a b] a] c]` computed against an unknown subject (because this is a gate).  It has two children, the `b` `[0 1]` and the `c` which is much longer.  Rule Eight is a sugar formula which essentially says, run `*[a b]` and then make that the head of a new subject, then compute `c` against that new subject.  `[0 1]` grabs the first argument of the `sample` in the `payload`, which is represented in Hoon by `a=@`.

The main formula is then the body of the gate.  It's another Rule Eight, this time to calculate the `b=0` line of the Hoon.

There's a Rule One, or constant reduction to return the bare value resulting from the formula.

Then one more Rule Eight (the last one!).  This one creates the default subject for the trap \texttt{\$}; this is implicit in Hoon.

Next, a Rule Six.  This is an `if`/`then`/`else` clause, so we expect a test and two branches.

- The test is calculated with Rule Five, an equality test between the address `30` of the subject and the increment of the `sample`.  In Hoon, `=(a +(b))`.

- The `[0 6]` returns the `sample` address.

- The other branch is a Rule Nine reboot of the subject via Rule Ten.  Note the `[4 0 6]` increment of the `sample`.

Finally, Rule Nine is invoked with `[9 2 0 1]`, which grabs a particular arm of the subject and executes it.

Contrast the built-in `++dec` arm:

```nock
> !=((dec 1))
[8 [9 2.398 0 1.023] 9 2 10 [6 7 [0 3] 1 1] 0 2]
```

for which the Hoon is:

```hoon
++  dec
  |=  a=@
  ?<  =(0 a)
  =+  b=0
  |-  ^-  @
  ?:  =(a +(b))  b
  $(b +(b))
```

Scan for pieces you recognize:  the beginning of a cell is frequently the rule being applied.

In tall form,

```hoon
[8
  [9
    [2.398 [0 1.023]]
  ]
  [9 2
       [10
         [6 7 [0 3] 1 1]
         [0 2]
       ]
  ]
]
```

What's going on with the above \texttt{++dec} is that the Arvo-shaped subject is being addressed into at `2.398`, then some internal Rule Nine/Ten/Six/Seven processing happens.

\section{Kelvin versioning}
\labsec{kelvin}

Each version of Nock

telescopic versioning

\section{Exercises}

Compose a Nock interpreter in a language of your choice.  (These aren't full Arvo interpreters, of course, since you don't have the Hoon, \zuse, and vane subject present.)

\setchapterpreamble[u]{\margintoc}
\chapter{Elements of Hoon}
\labch{hoonelemens}


\section{Reading the Runes}
\labsec{reading}

The goals of this section are for you to be able to:

\begin{enumerate}
  \item  Identify Hoon runes and children in both inline and long-form syntax.
  \item  Trace a short Hoon expression to its final result.
  \item  Produce output as a side effect using the \sigpam~rune.
\end{enumerate}

\marginnote[2mm]{
Although not a compiled language, the binary-tree structure of Hoon can lead to fairly involved programs which are difficult to type and parse as directly as some other languages afford.  We instead encourage you to use one of three methods to run Hoon programs:

\begin{enumerate}
  \item  The Dojo REPL, which offers some convenient shortcuts to modify the subject for subsequent commands.
  \item  A tight loop of text editor and running fakezod.
  \item  The online interactive sandbox at \url{https://approaching-urbit.com}{\texttt{https://approaching-urbit.com}}
\end{enumerate}
}

For the first several exercises, we will suggest that you utilize one of these methods in particular so that you get a feel for how each works.  After you are more comfortable working with Hoon code on Urbit, we will refrain.


The terminology used is often unfamiliar.  Sometimes this means that you are dealing with a truly new concept (and overloading an older word like "subroutine" or "function" would obfuscate), and sometimes you are dealing with an internal aspect that doesn't really map well to other systems.  The strangeness can be frustrating.  The strangeness can make concepts fresh again.  You'll encounter both as you move ahead.


Each rune accepts at least one child, except for \pzapzap.

\section{Irregular Forms}
\labsec{irregular}

Many runes in common currency are not written in their regular form (tall or wide), but rather using syntactic sugar as irregular.

For instance, \pcenhep~is most frequently written using parentheses \texttt{()} which permits a Lisp-like calling syntax:

\begin{lstlisting}
(add 1 2)
\end{lstlisting}

is equivalent to

\begin{lstlisting}
%-  add  [1 2]
\end{lstlisting}

is also equivalent to

\begin{lstlisting}
%-(add [1 2])
\end{lstlisting}



Hoon parses to an abstract syntax tree (AST), which includes cleaning up all of the sugar syntax and non-primitive runes.  To see the AST of any given Hoon expression, use \pzapcom.

\begin{lstlisting}[style=nonumbers]
> !,(*hoon TODO)
TODO
\end{lstlisting}

\section{Nouns}
\labsec{nouns}

All values in Urbit are nouns, meaning either atoms or cells.  An atom is an unsigned integer.  A cell is a pair of nouns.  Since all values are ultimately integers, we need a way to tell different "kinds" (or applications) of integers apart.  Enter auras.

\subsection{Atoms}
\labsec{atoms}

\marginnote[2mm]{
For what it may be worth, having all integers isn't that different from any other digital machine, built on binary numbers.  These all derive ultimately from \href{https://en.wikipedia.org/wiki/Gödel_numbering}{Gödel numbering} as introduced by Gödel in the proof of his famous incompleteness theorems.  Urbit makes this about as apparent as C does (via \texttt{union}, for instance), but it's first-order accessible via the Dojo REPL.
}
Atoms have auras which are tagged types.  In other words, an aura is a bit of metadata Hoon attaches to a value which tells Urbit how you intend to use a number.  (Of course, ultimately an aura is itself an integer as well!)  The default aura for a value is \patud, unsigned decimal, but of course there are many more.  Aura operations are extremely convenient for converting between representations.  They are also used to enforce type constraints on atoms in expressions and gates.

For instance, to a machine there is no fundamental difference between binary $\ub 1101\,1001$, decimal $217$, and hexadecimal $\ux d9$.  A human coder recognizes them as different encoding schemes and associates tacit information with each:  an assembler instruction, an integer value, a memory address.  Hoon offers two ways of designating values with auras:  either directly by the formatting of the number (such as \texttt{0b1101.1001}) or using the irregular syntax \texttt{`@`}:

\begin{lstlisting}
0b1101.1001
`@ud`0b1101.1001  :: yields 217
`@ux`0b1101.1001  :: yields 0xd9
\end{lstlisting}

\begin{example}
Try the following auras.  See if you can figure out how each one is behaving.

\begin{lstlisting}
`@ud`0x1001.1111
`@ub`0x1001.1111
`@ux`0x1001.1111
`@p`0x1001.1111

`@ud`.1
`@ux`.1
`@ub`.1
`@sb`.1

`@p`0b1111.0000.1111.0000

`@ta`'hello'
`@ud`'hello'
`@ux`'hello'
`@uc`'hello'
`@sb`'hello'
`@rs`'hello'
\end{lstlisting}
\end{example}

\marginnote[2mm]{For a full table of auras, see Appendix~\ref{app:auras}.}
The atom/aura system represents all simple data types in Hoon:  dates, floating-point numbers, text strings, Bitcoin addresses, and so forth.  Each value is represented in least-significant byte (LSB) order; for instance, a text string may be deconstructed as follows:

\begin{tabular}{l}
  \texttt{'Urbit'} \\
  \texttt{0b111.0100.0110.1001.0110.0010.0111.0010.0101.0101} \\
  \texttt{0b111.0100} \texttt{0b110.1001} \texttt{0b110.0010} \\ \texttt{0b111.0010} \texttt{0b101.0101} \\
  $116\;105\;98\;114\;85$ (ASCII characters) \\
  t i b r U
\end{tabular}

Note in the above that leading zeroes are always stripped.  Since each atom is an integer, there is no way to distinguish $0$ from $00$ from $000$ etc.

In this vein, it's worth mentioning that Dojo automatically parses any typed input and disallows invalid representations.  This can lead to confusion until you are accustomed to the type signatures; for instance, try to type \texttt{0b0001} into Dojo.

\paragraph{Operators}.  Hoon has no primitive operators.  Instead, aura-specific functions or \emph{gates} are used to evaluate one or more atoms to produce basic arithmetic results.  Gate names are conventionally prefixed with \texttt{++} which designates them as \emph{arms} of a \emph{core}.  (More on this terminology in Section~\ref{TODO}.)  Some gates operate on any input atom auras, while others enforce strict requirements on the types they will accept.  Gates are commonly invoked using a Lisp-like syntax and a reverse-Polish notation (RPN), with the operator first followed by the first and second (and following) operands.

\begin{tabular}{lll}
  Operation & Function & Example \\ \hline \\
  Addition & \texttt{++add} & \texttt{(add 1 2)} $\rightarrow$ \texttt{3} \\
  Subtraction & \texttt{++sub} & \texttt{(sub 4 3)} $\rightarrow$ \texttt{1} \\
  Multiplication & \texttt{++mul} & \texttt{(mul 5 6)} $\rightarrow$ \texttt{30} \\
  Division & \texttt{++div} & \texttt{(div 8 2)} $\rightarrow$ \texttt{4} \\
  Modulus/Remainder & \texttt{++mod} & \texttt{(mod 12 7)} $\rightarrow$ \texttt{5} \\
\end{tabular}

Following Nock's lead, Hoon uses loobeans ($0$ = $\textrm{true}$) rather than booleans for logical operations.  Loobeans are written \yes~for $\textrm{true}$, $0$, and \no~for $\textrm{false}$, $1$.

\begin{tabular}{lll}
  Operation & Function & Example \\ \hline \\
  Greater than & \texttt{++gth} & \texttt{(gth 5 6)} $\rightarrow$ \no \\
  Greater than or equal to & \texttt{++gte} & \texttt{(gte 5 6)} $\rightarrow$ \no \\
  Less than & \texttt{++lth} & \texttt{(lth 5 6)} $\rightarrow$ \yes \\
  Less than or equal to & \texttt{++lte} & \texttt{(lte 5 6)} $\rightarrow$ \yes \\
  Equals & \texttt{=} & \texttt{=(5 5)} $\rightarrow$ \yes \\
  Logical \texttt{AND}, $^TODO$ & \texttt{\&} & \texttt{\&(\%.y \%.n)} $\rightarrow$ \no \\
  Logical \texttt{OR}, $v$ & \texttt{|} & \texttt{|(\%.y \%.n)} $\rightarrow$ \yes \\
\end{tabular}

Operators can be grouped for precedence; \eg,

$$
TODO
$$

\begin{lstlisting}
\&()
\end{lstlisting}

The Hoon standard library, largely in \zuse, further defines bitwise operations, arithmetic for integers and floating-point values (half-width, single-precision, double-precision, and quadruple-precision), TODO
These are introduced incidentally as necessary and listed in more detail in Appendix~\ref{stdlib}.

\subsection{Cells}
\labsec{cells}



binary tree format
flattened representation/convention

\marginnote[2mm]{Binary trees are explained in more detail in Section~\ref{nock0}.}

Hoon values are addressed as elements in a binary tree.


Finally, the most general mold is \texttt{*} which simply matches any noun—and thus anything in Hoon at all.

\section{Hoon as Nock Macro}
\labsec{macro}

The point of employing Hoon is, of course, that Hoon compiles to Nock.  Rather than even say \emph{compile}, however, we should really just say Hoon is a \emph{macro} of Nock.  Each Hoon rune, data structure, and effect corresponds to a well-defined Nock primitive form.  We may say that Hoon is to Nock as C is to assembler, except that the Hoon-to-Nock transformation is completely specified and portable.  Hoon is ultimately defined in terms of Nock; many Hoon runes are defined in terms of other more fundamental Hoon runes, but all runes parse unambiguously to Nock expressions.


We call Hoon's data type specifications \emph{molds}.  Molds are more general than atoms and cells, but these form particular cases.  Hoon uses molds as a way of matching Nock tree structures (including Hoon metadata tags such as auras).



\marginnote[2mm]{
Be careful to not confuse \irrtis, which evaluates to \dottis, with the various \wut~runes like \wuttis.
}

\section{Key Data Structures}
\labsec{keydata}

\subsection{Lists}
\labsec{lists}

\subsection{Text}
\labsec{text}

\marginnote[2mm]{Both cords and tapes are casually referred to as strings.}

Hoon recognizes two basic text types:  the \emph{cord} or \patt~and the \emph{tape}.  Cords are single atoms containing the text as UTF-8 bytes interpreted as a single stacked number.  Tapes are lists of individual one-element cords.

\marginnote[2mm]{Lists are null-terminated, and thus so are tapes.}

Cords are useful as a compact primary storage and data transfer format, but frequently parsing and processing involves converting the text into tape format.  There are more utilities for handling tapes, as they are already broken up in a legible manner.

\begin{lstlisting}
++  trip
  |=  a=@  ^-  tape
  ?:  =(0 (met 3 a))  ~
  [^-(@ta (end 3 1 a)) $(a (rsh 3 1 a))]
\end{lstlisting}


For instance, \texttt{trip} converts a cord to a tape; \texttt{crip} does the opposite.

%\marginnote[2mm]{
%\begin{lstlisting}
%++  trip
%  |=  a/@  ^-  tape
%  ?:  =(0 (met 3 a))  ~
%  [^-(@ta (end 3 1 a)) $(a (rsh 3 1 a))]
%\end{lstlisting}
%\texttt{++met}, \texttt{++end}, and \texttt{++rsh} are bitwise manipulation gates.
%}

\marginnote[2mm]{
%\begin{lstlisting}
\texttt{++  crip  |=(a=tape `@t`(rap 3 a))}
%\end{lstlisting}

\texttt{++rap} assembles the list interpreted as cords with block size of $2^3$ (in this case).
}


All text in Urbit is UTF-8 (\emph{a fortiori} ASCII).  The \patc~UTF-32 aura is only used by \dill and Hood (the Dojo terminal agent).


\subsection{Cores, Gates, Doors}
\labsec{doors}

> anyway you can explicitly set the sample in an iron core
but you can't use it with +roll
New messages below
11:54 (~master-morzod)
\gold~is the default, read/write everything; \iron~is for functions (write to the sample with a contravariant nest check), \lead~is "hide the whole payload", \zinc~completes the matrix
but has probably never been used
\iron~lets you refer to a typed gate (without wetness), without depending on all the details of the subject it was defined against
\lead~lets you export a library interface but hide the implementation details

\subsection{Molds}
\labsec{molds}

\subsection{Maps, Sets, Tree}
\labsec{macro}

\section{Generators}
\labsec{generators}

\subsection{Naked Generators}
\labsec{naked}

\subsection{\say~generators}
\labsec{say}

\subsection{\ask~generators}
\labsec{ask}

\section{Libraries}
\labsec{libraries}

\section{Unit Tests}
\labsec{unittests}

\section{Building Code}
\labsec{building}

\setchapterpreamble[u]{\margintoc}
\chapter{Advanced Hoon}
\labch{advancedhoon}


\section{Cores}
\labsec{cores}

\subsection{Variadicity}
\labsec{variadicity}

\subsection{Genericity}
\labsec{genericity}

\section{Molds}
\labsec{molds2}

As we saw when discussing auras, molds are the most general category of type in Hoon.

\marginnote[2mm]{hard/soft atoms, seeing atoms, etc. TODO}

\subsection{Polymorphism}
\labsec{polymorphism}

\section{Rune Families}
\labsec{runefamilies}

\section{Marks and Structures}
\labsec{marks}

\section{Helpful Tools}
\labsec{tools}

\marginnote[2mm]{See also Section~\ref{factories} which discusses common "factory patterns" in subject-oriented programming.}

\section{Deep Dives}
\labsec{deepdives}

\subsection{Text Stream Parsing}
\labsec{text}

\subsection{JSON Parsing}
\labsec{json}

\subsection{HTML/XML Parsing}
\labsec{sail}


\pagelayout{wide} % No margins
\addpart{System Development}
\pagelayout{margin} % Restore margins

\setchapterpreamble[u]{\margintoc}
\chapter{The Kernel}
\labch{kernel}


\section{Arvo}
\labsec{arvo}

Arvo is essentially an event handler which can coordinate and dispatch messages between vanes as well as emit \unix~events to the underlying (presumed Unix-compatible) host OS.  Arvo does not carry out several tasks specific to the machine hardware, such as memory allocation, system thread management, and hardware- or firmware-level operations.  These are left to the king and serf, or the daemon processes which together run Arvo.  Collectively, the system-level instrumentation of Arvo is described in Chapter~\ref{support}.

\subsection{\zuse~and \lull}
\labsec{zuse}

\section{\ames, A Network}
\labsec{ames}

\section{\behn, A Timer}
\labsec{behn}

\section{\clay, A File System}
\labsec{clay}

\subsection{\ford, A Build System}
\labsec{ford}

\subsection{Scrying}
\labsec{scry}

\subsection{Marks and conversions}
\labsec{marks2}

\section{\dill, A Terminal driver}
\labsec{dill}

\section{\eyre~and \iris, Server and Client Vanes}
\labsec{eyre}

\section{\jael, Secretkeeper}
\labsec{jael}

\section{Azimuth, Address Space Management}
\labsec{azimuth2}

\section{The Hoon Parser}
\labsec{hoonparse}

\setchapterpreamble[u]{\margintoc}
\chapter{Userspace}
\labch{Userspace}


\section{\gall, A Runtime Agent}
\labsec{gall}

\marginnote[2mm]{
Modern \gall~is sometimes called "static Gall," in contrast to an earlier specification "dynamic Gall."  Dynamic Gall did not specify the arms and permitted each agent its own structure; in practice, this proved to be difficult for programmers to maintain in a consistent manner, leading to code refactors and maintenance of defunct arms for backwards compatibility of agents.
}


\subsection{Factory Patterns}
\labsec{factories}

\section{Deep Dives in \gall}

Each of the following case studies is drawn from published code, most of it incorporated into the Urbit userspace.  In some cases, the original code uses conventions we have not yet introduced; we have simplified these to rely on the runes introduced in the main text through Chapter~\ref{advancedhoon}.

\subsection{Chat CLI}
\labsec{wchat}

\subsection{Drum and Helm}
\labsec{wdrum}

\subsection{Bitcoin API}
\labsec{wbtc}

\subsection{Bots}
\labsec{wbots}

\section{Threading with Spider}

\section{Urbit API}

\section{Deep Dives with Urbit API}

\subsection{Time (Clock)}
\labsec{wtime}

\subsection{Publish}
\labsec{wpublish}

\subsection{\graphstore}
\labsec{wgraph}

\setchapterpreamble[u]{\margintoc}
\chapter{Supporting Urbit}
\labch{su}


\section{Booting and Pills}
\labsec{su:boot}

\section{\unix~Events}
\labsec{su:unix}

\section{Nock Virtual Machines}
\labsec{su:nockvm}

\subsection{\mock}
\labsec{su:mock}

\section{King and Serf Daemons}
\labsec{su:kingserf}

\subsection{Vere (Reference C Implementation)}
\labsec{su:vere}

\subsection{King Haskell (Haskell Implementation)}
\labsec{su:haskell}

\subsection{Jaque (JVM Implementation)}
\labsec{su:jaque}

\section{Jetting}
\labsec{su:jetting}

\subsection{Jet matching and the dashboard}
\labsec{su:dashboard}

\setchapterpreamble[u]{\margintoc}
\chapter{Concluding Remarks}
\labch{conclusion}


\section{Booting and Pills}
\labsec{boot}


\appendix % From here onwards, chapters are numbered with letters, as is the appendix convention

\pagelayout{wide} % No margins
\addpart{Appendix}
\pagelayout{margin} % Restore margins

\setchapterpreamble[u]{\margintoc}
\chapter{Appendices}
\labch{appendix}


\section{Comprehensive table of Hoon runes}
\labsec{ap:runes}

These runes are up-to-date as of Hoon \texttt{\%140}.  Runes are ordered by alphabetical pronunciation of Urbit-standard aural ASCII.  Standard definitions (such as \texttt{+\$map} and \texttt{+\$unit}) are defined in Section~\ref{he:structures}.  Of those remaining, the most common is \texttt{+\$hoon}, which is used as a shorthand for any valid Hoon expression.  A \texttt{+\$term} is TODO tome spec value

Digraphs

\begin{table}
  \caption{Aural ASCII Pronunciation}
  \label{}
  \begin{tabular}{clclclcl}
    \texttt{\textasciitilde} & “sig” &
    \texttt{!} & “zap” &
    \texttt{\@} & “pat” \\
  \end{tabular}
\end{table}

\subsection{\texttt{|} “bar”:  Core Definition}
\labsec{ap:bar}

\pbar~runes produce cores.

\begin{tabular}{l}
\hline \\ \hline \\
\textbf{\barbuc\texttt{~(lest term) spec}}
\\
\pbarbuc~produces a mold from a TODO.
\\
\begin{lstlisting}[style=nonumbers]
TODO
\end{lstlisting}
\\
\begin{lstlisting}[style=nonumbers]
::  $mk-item: constructor for +ordered-map item type
++  mk-item  |$  [key val]  [key=key val=val]
\end{lstlisting}
\\
\hhline{=} \\
\textbf{\barcab\texttt{~spec alas (map term tome)}}
\\ \hline
\\
\pbarcab produces a door (a core with sample) given XYZ
\\
\hline \\ \hline \\
\textbf{\barcen\texttt{~(unit term) (map term tome)}}
\\
produces a core (battery and payload)
\\
\hhline{=} \\
\textbf{\barcol\texttt{[hoon hoon]}}
\\
produces a gate with a custom sample
\hhline{=} \\
\textbf{\bardot\texttt{~hoon}}
\\
produces a trap (a core with one arm)
\hhline{=} \\
\textbf{\barhep\texttt{~hoon}}
\\
produces a trap (a core with one arm) and evaluates it
\hhline{=} \\
\textbf{\barket\texttt{~hoon (map term tome)}}
\\
produces a core whose battery includes a \$ arm and computes the latter
\hhline{=} \\
\textbf{\barpat\texttt{~(unit term) (map term tome)}}
\\
produces a wet core (battery and payload)
\hhline{=} \\
\textbf{\barsig\texttt{~[spec value]}}
\\
produces an iron gate
\hhline{=} \\
\textbf{\bartar\texttt{~[spec value]}}
\\
produces a wet gate (a one-armed core with sample)
\hhline{=} \\
\textbf{\bartis\texttt{~[spec value]}}
\\
produces a dry gate (a one-armed core with sample)
\\
\hhline{=} \\
\textbf{\barwut\texttt{~hoon}}
\\
produces a lead trap

\end{tabular}

%------------------------------------------------------------------------------%
\subsection{\texttt{\S} “buc”:  Mold Definition}
\labsec{ap:buc}

\pbuc~runes produce mold definitions.

%------------------------------------------------------------------------------%
\subsection{\texttt{\%} “cen”:  Core Evaluation}
\labsec{ap:cen}

\pcen~runes evaluate cores (similar to function calls in other languages).

\begin{lstlisting}
(~(rad og eny) 5)
(rad:og 5)  :: TODO figure out tersest working equivalent here
\end{lstlisting}

%------------------------------------------------------------------------------%
\subsection{\texttt{:} “col”:  Cell Construction}
\labsec{ap:col}

\pcol~runes produce cells of various structures.

%------------------------------------------------------------------------------%
\subsection{\texttt{.} “dot”:  Nock Evaluation}
\labsec{ap:dot}

\pdot~runes directly evaluate as Nock expressions.

%------------------------------------------------------------------------------%
\subsection{\texttt{\^} “ket”:  Core Typecasting}
\labsec{ap:ket}

\pket~runes are used to alter cores.

%------------------------------------------------------------------------------%
\subsection{\texttt{\textasciitilde} “sig”:  Hinting}
\labsec{ap:sig}

\psig~runes represent directives to the runtime.

%------------------------------------------------------------------------------%
\subsection{\texttt{;} “mic”:  Macro}
\labsec{ap:mic}

\mic~runes produce calling structures (mainly monadic binds) and XML elements.

%------------------------------------------------------------------------------%
\subsection{\texttt{=} “tis”:  Subject Alteration}
\labsec{ap:tis}

\tis~runes modify the subject.

%------------------------------------------------------------------------------%
\subsection{\texttt{?} “wut”:  Comparison}
\labsec{ap:wut}

\wut~runes compare subtrees.

%------------------------------------------------------------------------------%
\subsection{\texttt{!} “zap”:  Wildcard}
\labsec{ap:zap}

\zap~runes perform a miscellaneous set of auxiliary operations.

%%%%%%%%%%%%%%%%%%%%%%%%%%%%%%%%%%%%%%%%%%%%%%%%%%%%%%%%%%%%%%%%%%%%%%%%%%%%%%%%
\section{Hoon versions}
\labsec{ap:hoonversions}

%------------------------------------------------------------------------------%

%%%%%%%%%%%%%%%%%%%%%%%%%%%%%%%%%%%%%%%%%%%%%%%%%%%%%%%%%%%%%%%%%%%%%%%%%%%%%%%%
\section{Nock versions}
\labsec{ap:nockversions}

%%%%%%%%%%%%%%%%%%%%%%%%%%%%%%%%%%%%%%%%%%%%%%%%%%%%%%%%%%%%%%%%%%%%%%%%%%%%%%%%
\section{Hoon comparison with other languages}
\labsec{ap:comparison}

%%%%%%%%%%%%%%%%%%%%%%%%%%%%%%%%%%%%%%%%%%%%%%%%%%%%%%%%%%%%%%%%%%%%%%%%%%%%%%%%
\section{\zuse/\lull~versions}
\labsec{ap:zuseversions}

%%%%%%%%%%%%%%%%%%%%%%%%%%%%%%%%%%%%%%%%%%%%%%%%%%%%%%%%%%%%%%%%%%%%%%%%%%%%%%%%
\section{Textbook changelog}
\labsec{ap:changelog}


%----------------------------------------------------------------------------------------

\backmatter % Denotes the end of the main document content
\setchapterstyle{plain} % Output plain chapters from this point onwards

%----------------------------------------------------------------------------------------
%	BIBLIOGRAPHY
%----------------------------------------------------------------------------------------

% The bibliography needs to be compiled with biber using your LaTeX editor, or on the command line with 'biber main' from the template directory

\defbibnote{bibnote}{Here are the references in citation order.\par\bigskip} % Prepend this text to the bibliography
\printbibliography[heading=bibintoc, title=Bibliography, prenote=bibnote] % Add the bibliography heading to the ToC, set the title of the bibliography and output the bibliography note

%----------------------------------------------------------------------------------------
%	NOMENCLATURE
%----------------------------------------------------------------------------------------

% The nomenclature needs to be compiled on the command line with 'makeindex main.nlo -s nomencl.ist -o main.nls' from the template directory

\nomenclature{$c$}{Speed of light in a vacuum inertial frame}
\nomenclature{$h$}{Planck constant}

\renewcommand{\nomname}{Notation} % Rename the default 'Nomenclature'
\renewcommand{\nompreamble}{The next list describes several symbols that will be later used within the body of the document.} % Prepend this text to the nomenclature

\printnomenclature % Output the nomenclature

%----------------------------------------------------------------------------------------
%	INDEX
%----------------------------------------------------------------------------------------

% The index needs to be compiled on the command line with 'makeindex main' from the template directory

\printindex % Output the index

%----------------------------------------------------------------------------------------
%	BACK COVER
%----------------------------------------------------------------------------------------

% If you have a PDF/image file that you want to use as a back cover, uncomment the following lines

%\clearpage
%\thispagestyle{empty}
%\null%
%\clearpage
%\includepdf{cover-back.pdf}

%----------------------------------------------------------------------------------------

\end{document}
