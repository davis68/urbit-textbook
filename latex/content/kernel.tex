\setchapterpreamble[u]{\margintoc}
\chapter{The Kernel}
\labch{kr}


\section{Arvo}
\labsec{kr:arvo}

We could do worse than to start our exploration of the Urbit kernel than to quote the Whitepaper itself on the subject of Arvo:

\begin{quote}
The fundamental unit of the Urbit kernel is an event called a \texttt{+\$move}.  Arvo is primarily an event dispatcher between moves created by vanes.  Each move therefore has an address and other structured information.  (\cite{Yarvin2017})
\end{quote}

Every vane defines its own structured events (\texttt{+\$move}s).  Each unique kind of structured event has a unique, frequently whimsical, name.  This can make it challenging to get used to how a particular vane behaves.

\marginnote[2mm]{The system-level instrumentation of Arvo is described in Chapter~\ref{support}.}

Arvo is essentially an event handler which can coordinate and dispatch messages between vanes as well as emit \unix~events (side effects) to the underlying (presumed Unix-compatible) host OS.  Arvo as hosted OS does not carry out any tasks specific to the machine hardware, such as memory allocation, system thread management, and hardware- or firmware-level operations.  These are left to the \emph{king} and \emph{serf}, the daemon processes which together run Arvo.

Arvo is architected as a state machine, the deterministic end result of the event log.  We need to examine Arvo from two separate angles:

\begin{enumerate}
  \item  Event processing engine and state machine (vane coordinator)
  \item  Standard noun structure (“Arvo-shaped noun”)
\end{enumerate}

\subsection{Event Processing Engine}
\labsec{kr:arvo-state}

\begin{quote}
A vanilla event loop scales poorly in complexity.  A system event is the trigger for a cascade of internal events; each event can schedule any number of future events.  This easily degenerates into “event spaghetti.”  Arvo has “structured events”; it imposes a stack discipline on event causality, much like imposing subroutine structure on \texttt{goto}s.  (\cite{Yarvin2017})
\end{quote}

Arvo events are known as \texttt{++move}s, containing metadata and data.  Arvo recognizes four types of moves:

\begin{enumerate}
  \item  \pass~events are forward calls from one vane to another (or back to itself, occasionally), and
  \item  \give~events are returned values and move back along the calling duct.
  \item  \slip~events [TODO think of like ref counting?]
  \item  \unix~events communicate from Arvo to the underlying binary in such a way as to emit an external effect (an \ames~network communication, for instance, or text input and output).
\end{enumerate}

show structure of a card

"Each vane defines a protocol for interacting with other vanes (via Arvo) by defining four types of cards: tasks, gifts, notes, and signs."
"In other words, there are only four ways of seeing a move: (1) as a request seen by the caller, which is a note. (2) that same request as seen by the callee, a task. (3) the response to that first request as seen by the callee, a gift. (4) the response to the first request as seen by the caller, a sign."
(TODO move trace work here)

Without reference to the particular content of vanes, let us briefly diagram a “move trace”, or examination of how an event generated by a vane produces results via Arvo.

TODO remote call via \ames?

\begin{quote}
"An interrupted event never happened.  The computer is deterministic; an event is a transaction; the event log is a log of successful transactions. In a sense, replaying this log is not Turing complete. The log is an existence proof that every event within it terminates.  (\cite{Yarvin2017})
\end{quote}

\subsection{Standard Noun Structure}
\labsec{kr:arvo-noun}

Arvo defines five standard arms for vanes and the binary runtime to use:

\begin{enumerate}
  \item  \texttt{++peek} grants read-only access to \clay; this is called a \emph{scry}.
  \item  \texttt{++poke} accepts \texttt{++move}s and processes them; this is the only arm that actually alters Arvo's state.
  \item  \texttt{++wish} accepts a core and parses it against \zuse.
  \item  \texttt{++come} and \texttt{++load} are used in kernel upgrades, allowing Arvo to update itself in-place.
\end{enumerate}


\href{Arvo tutorial}{https://urbit.org/docs/arvo/overview/}

\subsection[\zuse~\&~\lull]{\zuse~and \lull}
\labsec{zuse}

\zuse~and \lull~define common structures and library functions for Arvo.

subject wrapped

\section{Arvo Vanes}
\labsec{kr:vanes}

Each vane has a characteristic structure which identifies it as a vane to Arvo and allows it to handle moves consistently.

\section[\ames]{\ames, A Network}
\labsec{kr:ames}

In a sense, \ames~is the operative definition of an urbit on the network.  That is, from outside of one's own urbit, the only specification that must be hewed to is that \ames~behaves a certain way in response to events.

\ames~implements a system expecting—and delivering—guaranteed one-time delivery.  This derives from an observation by \citeauthor{Yarvin2016}~in the Whitepaper:  "bus v. commands whatever"

UDP packet structure

network events
acks \& nacks

\subsection{Scrying into \ames}
\labsec{kr:a:scry}

\ames~scry

\begin{table}[h!]
  \begin{center}
    \caption{\ames~\dotket~Calls.}
    \label{ha:ames}
    \begin{tabular}{lll}
      Symbol & Meaning & Example \\
      \hline \\
      \texttt{\%x} & Get ship and peer information:  protocol version, peers, ship state, etc. & \\
    \end{tabular}
  \end{center}
\end{table}


\section[\behn]{\behn, A Timer}
\labsec{behn}

\behn~is a simple vane that promises to emit events after—but never before—their timestamp.  This guarantee

As the shortest vane, we commend \behn~to the student as an excellent subject for a first dive into the structure of a vane.

\behn~maintains an event handler and a state.

Any task may have one of the following states:

\begin{lstlisting}
%born  born:event-core
%rest  (rest:event-core date=p.task)
%drip  (drip:event-core move=p.task)
%huck  (huck:event-core syn.task)
%trim  trim:event-core
%vega  vega:event-core
%wait  (wait:event-core date=p.task)
%wake  (wake:event-core error=~)
\end{lstlisting}

\subsection{Scrying into \behn}
\labsec{kr:b:scry}

\begin{table}[h!]
  \begin{center}
    \caption{\behn~\dotket~Calls.}
    \label{ha:behn}
    \begin{tabular}{lll}
      Symbol & Meaning & Example \\
      \hline \\
      \texttt{\%x} & Get timers, timestamps, next timer to fire, etc. & \\
    \end{tabular}
  \end{center}
\end{table}

\section[\clay]{\clay, A File System}
\labsec{clay}

```
~tinnus-napbus
3:57 PM
what is the correct way to read a file on a remote ship? I've tried both warp and werp and I'm not getting a response, just messages in the target about clay something something indirect and then I get crash on fragment errors sometimes
with an %x care
~rovnys-ricfer
4:02 PM
warp should work, I think
make sure the file is actually there
i.e. can you do the same warp on the local ship?
```

\begin{itemize}
  \item  \texttt{aeon} is
  \item  \texttt{arch} is
  \item  \texttt{care} is the \clay~submode, defined in \lull.
  \item  \texttt{desk} is
  \item  \texttt{dome} is
  mark cast file
\end{itemize}


\subsection{Scrying into \clay}
\labsec{kr:c:scry}

\clay~has the most sophisticated scry taxonomy.

\begin{table}[h!]
  \begin{center}
    \caption{\clay~\dotket~Calls.}
    \label{ha:clay}
    \begin{tabular}{lll}
      Symbol & Meaning & Example \\
      \hline \\
      \texttt{\%a} & Expose file build into namespace. & \texttt{.\^{}(vase \%ca /\~zod/home/1/lib/generators/hoon} \\
      \texttt{\%b} & Expose mark build into namespace. & \\
      \texttt{\%c} & Expose cast build into namespace. & \\
      %\texttt{\%d} & & \\
      \texttt{\%e} & & \\
      \texttt{\%f} & & \\
      \texttt{\%p} & Get the permissions that apply at path. & \\
      \texttt{\%r} & Get the data at a node (like \texttt{\%x}) and wrap it in a vase. & \\
      \texttt{\%s} & produce yaki or blob for given tako or lobe & \\
      \texttt{\%t} & produce the list of paths within a yaki with :pax as prefix & \\
      \texttt{\%u} & Check for existence of a node at aeon. & \\
      \texttt{\%v} & Get the desk state at a specified aeon. & \\
      \texttt{\%w} & Get all cases referring to the same revision as the given case. & \\
      \texttt{\%x} & Get the data at a node. & \\
      \texttt{\%y} & Get the \texttt{arch} (directory listing) at a node. & \\
      \texttt{\%z} & Get a recursive hash of a node and its children. & \\
    \end{tabular}
  \end{center}
\end{table}

% https://github.com/urbit/urbit/blob/master/pkg/arvo/sys/vane/gall.hoon#L1538
% https://urbit.org/blog/ford-fusion/


\subsection[\ford]{\ford, A Build System}
\labsec{ford}

runes
%https://urbit.org/blog/ford-fusion/

\begin{table}[h!]
  \begin{center}
    \caption{\ford~Runes.}
    \label{ha:ford}
    \begin{tabular}{lll}
      Symbol & Meaning & Example \\
      \hline \\
      \texttt{/-} & Import structure file from \texttt{sur}. & \\
      \texttt{/+} & Import library file from \texttt{lib}. & \\
      \texttt{/=} & Import user-specified file. & \\
      \texttt{/*} & Import contents of file converted by mark. & \\
    \end{tabular}
  \end{center}
\end{table}

importing with \texttt{*} is w/o face, foo=bar


\subsection[Marks]{Marks and conversions}
\labsec{marks2}

\section[\dill]{\dill, A Terminal driver}
\labsec{dill}

\subsection{Scrying into \dill}
\labsec{kr:d:scry}

\dill~scrys are unusual, in that they are typically only necessary for fine-grained Arvo control of the display.  Even command-line apps instrumented with \shoe~do not call into \dill~commonly.  The only instance of use in the current Arvo kernel is in Herm, the terminal session manager.

\begin{table}[h!]
  \begin{center}
    \caption{\dill~\dotket~Calls.}
    \label{ha:dill}
    \begin{tabular}{lll}
      Symbol & Meaning & Example \\
      \hline \\
      \texttt{\%x} & Get the current line or cursor position of default session. & \\
    \end{tabular}
  \end{center}
\end{table}


\section[\eyre~\&~\iris]{\eyre~and \iris, Server and Client Vanes}
\labsec{eyre}

\subsection{Scrying into \eyre}
\labsec{kr:e:scry}

\eyre~

\begin{table}[h!]
  \begin{center}
    \caption{\jael~\dotket~Calls.}
    \label{ha:iris}
    \begin{tabular}{lll}
      Symbol & Meaning & Example \\
      \hline \\
      \texttt{\%x} & Get CORS etc TODO. & \\
      \texttt{\%\$} & Get . & \\  % TODO is this default subject?
    \end{tabular}
  \end{center}
\end{table}

\subsection{Scrying into \iris}
\labsec{kr:i:scry}

\begin{table}[h!]
  \begin{center}
    \caption{\iris~\dotket~Calls.}
    \label{ha:iris}
    \begin{tabular}{lll}
      Symbol & Meaning & Example \\
      \hline \\
      \texttt{\%\$} & Get . & \\  % TODO is this default subject?
    \end{tabular}
  \end{center}
\end{table}

\section[\jael]{\jael, Secretkeeper}
\labsec{jael}

\marginnote[2mm]{
\jael is named after Jael, the wife of Heber, who kept mum and slew fleeing enemy general Sisera in Judges 4.  It also puns on “jail”.
}

\jael~keeps secrets, the cryptographic keys that make it possible to securely control your Urbit.  Among other cryptographic facts, \jael~keeps track of the following:

\begin{enumerate}
  \item  Subscription to \texttt{\%azimuth-tracker}, the current state of the Azimuth PKI.
  \item  Initial public and private keys for the ship.
  \item  Public keys of all galaxies.
  \item  Record of Ethereum registration for Azimuth.
\end{enumerate}

\jael~weighs in as one of the shorter vanes, but is critical to Urbit as a secure network-first operating system.  \jael~is in fact the first vane loaded after \dill~when bootstrapping Arvo on a new instance.

\begin{lstlisting}

\end{lstlisting}


record the Ethereum block the public key is registered to,
record the URL of the Ethereum node used,
save the signature of the parent planet (if the ship is a moon),
load the initial public and private keys for the ship,
set the DNS suffix(es) used by the network (currently just arvo.network),
save the public keys of all galaxies,
set Jael to subscribe to %azimuth-tracker,
%slip a %init task to Ames, Clay, Gall, Dill, and Eyre, and %give an %init gift to Arvo, which then informs Unix that the initialization process has concluded.

~master-morzod
1:00 PM
the private key (ring) is 64 bytes, plus a tag byte
1:00
(met 3 sec:ex:(pit:nu:crub:crypto 512 eny))
65
the ship is up to 64 bits (or 128, if you want to support comet keypairs, which you could)
\begin{lstlisting}
=/  ,seed:jael  [who=`@p`0 lyf=1 sek=(bex 520) ~]  (met 3 (scot %uw (jam -)))
=/  ,seed:jael  [who=`@p`(bex 127) lyf=(bex 31) sek=(bex 520) ~]  (met 3 (scot %uw (jam -)))
\end{lstlisting}
likely lower and upper bounds


\href{https://urbit.org/docs/arvo/jael/jael-api/}{\jael~API Reference}



\subsection{Scrying into \jael}
\labsec{kr:j:scry}




\marginnote[2mm]{
Sometimes residual elements of vane development leak into release code and yield insight into how the kernel developers produce new vanes.  For instance, when a new version of \jael~was being developed, it was dubbed \texttt{kale} and used the scry path \texttt{\%k}.  This made it into a release version of \texttt{lib/ring} in Arvo 309 K.
}

\begin{table}[h!]
  \begin{center}
    \caption{\jael~\dotket~Calls.}
    \label{ha:jael}
    \begin{tabular}{lll}
      Symbol & Meaning & Example \\
      \hline \\
      \texttt{\%\$} & Get TODO. & \\
      \texttt{\%\$} & Get . & \\
    \end{tabular}
  \end{center}
\end{table}


\subsection{Launching a Moon}
\labsec{kr:j:moon}

It is instructive, as a brief examination of how \jael~works, to see how to launch and operate a moon (a 64-bit identity tied to a parent planet).  We begin our enquiry with an examination of Hood's \texttt{moon} generator.  This generator optionally accepts a particular moon name.  When run, \texttt{|moon} returns a private key appropriate to registering a new moon ship with the planet's \jael.  This key is then used to launch the moon ship using \jael to authenticate against the planet.

\begin{lstlisting}
~sampel-palnet:dojo> |moon ~fosfyr-dinler-sampel-palnet
moon: ~fosfyr-dinler-sampel-palnet
0wayzmv.OyFW6.FL6bt.GYKeZ.tPPfc.vJ7av.nFvA0.210dK.iAHiq.cSOGh.MCTuz.RiDIV.A3Pd8.pPZcV.djykA.kzrxV.P6M0s.~oZge.Nsjpr.K-WAh.QoUOP
\end{lstlisting}

Let's take a look at \texttt{|moon}:

\begin{lstlisting}[caption={\texttt{gen/hood/moon.hoon}, 🄯Tlon}]
::  Create a private key-file for a random (or specified) moon
::
::::  /hoon/moon/hood/gen
  ::
/-  *sole
/+  *generators
::
::::
  ::
:-  %say
|=  $:  [now=@da eny=@uvJ bec=beak]
        arg=?(~ [mon=@p ~])
        public-key=pass
    ==
:-  %helm-moon
^-  (unit [=ship =udiff:point:jael])
=*  our  p.bec
=/  ran  (clan:title our)
?:  ?=([?(%earl %pawn)] ran)
  %-  %-  slog  :_  ~
      leaf+"can't create a moon from a {?:(?=(%earl ran) "moon" "comet")}"
  ~
=/  mon=ship
  ?^  arg
    mon.arg
  (add our (lsh 5 (end 5 (shaz eny))))
=/  seg=ship  (sein:title our now mon)
?.  =(our seg)
  %-  %-  slog  :_  ~
      :-  %leaf
      "can't create keys for {(scow %p mon)}, which belongs to {(scow %p seg)}"
  ~
=/  =pass
  ?.  =(*pass public-key)
    public-key
  =/  cub  (pit:nu:crub:crypto 512 (shaz (jam mon life=1 eny)))
  =/  =seed:jael
    [mon 1 sec:ex:cub ~]
  %-  %-  slog
      :~  leaf+"moon: {(scow %p mon)}"
          leaf+(scow %uw (jam seed))
      ==
  pub:ex:cub
`[mon *id:block:jael %keys 1 1 pass]
\end{lstlisting}

TODO

The \texttt{\%helm-moon} tag tells Helm to invoke its \texttt{++poke-moon} arm and cycle the keys of the moon (or create them).

\begin{lstlisting}[style=nonumbers,
                   caption={\texttt{lib/hood/helm.hoon}, line 211}]
%helm-moon             =;(f (f !<(_+<.f vase)) poke-moon)
\end{lstlisting}

\begin{lstlisting}[style=nonumbers,
                   caption={\texttt{lib/hood/helm.hoon}, lines 75–80}]
++  poke-moon                                        ::  rotate moon keys
  |=  sed=(unit [=ship =udiff:point:jael])
  =<  abet
  ?~  sed
    this
  (emit %pass / %arvo %j %moon u.sed)
\end{lstlisting}

where \texttt{++emit} is the Helm event processor:

\begin{lstlisting}[style=nonumbers,
                   caption={\texttt{lib/hood/helm.hoon}, line 21–24, 29}]
|=  [=bowl:gall sat=state]
=|  moz=(list card)
|%
++  this  .
++  emit  |=(card this(moz [+< moz]))
\end{lstlisting}

\marginnote[2mm]{
Azimuth points underwent several shifts in nomenclature.  Although currently organized into the intuitive hierarchical galaxy/star/planet/moon/comet syntax, the original scheme involved \duke/\earl/\lord/\pawn/\wolf.  Sometimes this terminology remains visible in internal kernel types such as \texttt{\%earl} for moon; this is from the \czar/\king/\duke/\earl/\pawn~phase, which itself predates the carrier/cruiser/destroyer/yacht/submarine period.
}

When a new moon is launched as a ship, as part of the Arvo bootstrapping sequence the moon's status as a daughter point of the given planet is checked.  First, whether the sponsoring point is validly a moon (as \texttt{\%earl}); then whether said moon can in fact be validated as belonging to a planet we control (\texttt{sein:title} checks for sponsorship).  If both of these checks pass, then the private-key update event is issued using the \texttt{++new-event} arm of \jael.  The most salient part of this code, from a cryptographic standpoint, is \texttt{udiff}, a non-invertible \texttt{diff} or the new public key.

\marginnote[2mm]{
Note the \sig~call format to invoke an arm in a door, last line of this code snippet.  \texttt{new-event} is an arm in \texttt{su} (part of \jael) invoked along with the duct \texttt{hen}, the time \texttt{now}, the current public-key state \texttt{pki}, and the Ethernet node \texttt{etn}.
}

\begin{lstlisting}
::
::  update private keys
::
    %moon
  ?.  =(%earl (clan:title ship.tac))
    ~&  [%not-moon ship.tac]
    +>.$
  ?.  =(our (^sein:title ship.tac))
    ~&  [%not-our-moon ship.tac]
    +>.$
  %-  curd  =<  abet
  (~(new-event su hen now pki etn) [ship udiff]~:tac)
\end{lstlisting}

As for the rest—please see Section~\ref{su:boot} for more information on the boot sequence of a new ship.

A moon can scry into \jael

\begin{lstlisting}
::
    %earl
  ?.  ?=([@ @ ~] tyl)  [~ ~]
  ?.  =([%& our] why)
    [~ ~]
  =/  who  (slaw %p i.tyl)
  =/  lyf  (slaw %ud i.t.tyl)
  ?~  who  [~ ~]
  ?~  lyf  [~ ~]
  ?:  (gth u.lyf lyf.own.pki.lex)
    ~
  ?:  (lth u.lyf lyf.own.pki.lex)
    [~ ~]
  :: XX check that who/lyf hasn't been booted
  ::
  =/  sec  (~(got by jaw.own.pki.lex) u.lyf)
  =/  moon-sec  (shaf %earl (sham our u.lyf sec u.who))
  =/  cub  (pit:nu:crub:crypto 128 moon-sec)
  =/  =seed  [u.who 1 sec:ex:cub ~]
  ``[%seed !>(seed)]
\end{lstlisting}

(Now that you have a moon, what can you do with it?)


\section[Azimuth]{Azimuth, Address Space Management}
\labsec{azimuth2}

Urbit HD wallet


\paragraph{Comet keys}.  Comets do not have an associated Urbit HD wallet, and their keys work slightly differently.

> yeah thats how comet mining works. so you'd just put the private key you generated for a comet on the card, and this would be the ames DH exchange private key. i suppose you could still obfuscate it with a master ticket @q, by just picking a 128 bit hash, but it would be used differently than a normal azimuth master key, which is a @q used to derive a bunch of ethereum wallets private keys (and ultimately the initial network key, but that isnt required).
9:15
>and yeah whether a key works at the time it is mined is dependent on whether the routing node automatically assigned to the comet public key is currently working
>~lagrev-nocfep: the comets name is its public key


\section[Hoon Parser]{The Hoon Parser}
\labsec{hoonparse}
