\setchapterpreamble[u]{\margintoc}
\chapter{The Kernel}
\labch{kernel}


\section{Arvo}
\labsec{arvo}

Arvo is essentially an event handler which can coordinate and dispatch messages between vanes as well as emit \unix~events to the underlying (presumed Unix-compatible) host OS.  Arvo does not carry out several tasks specific to the machine hardware, such as memory allocation, system thread management, and hardware- or firmware-level operations.  These are left to the king and serf, or the daemon processes which together run Arvo.  Collectively, the system-level instrumentation of Arvo is described in Chapter~\ref{support}.

\subsection{\zuse~and \lull}
\labsec{zuse}

\section{\ames, A Network}
\labsec{ames}

\section{\behn, A Timer}
\labsec{behn}

\section{\clay, A File System}
\labsec{clay}

\subsection{\ford, A Build System}
\labsec{ford}

\subsection{Scrying}
\labsec{scry}

\subsection{Marks and conversions}
\labsec{marks2}

\section{\dill, A Terminal driver}
\labsec{dill}

\section{\eyre~and \iris, Server and Client Vanes}
\labsec{eyre}

\section{\jael, Secretkeeper}
\labsec{jael}

\section{Azimuth, Address Space Management}
\labsec{azimuth2}

\section{The Hoon Parser}
\labsec{hoonparse}
