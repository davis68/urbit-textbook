\setchapterpreamble[u]{\margintoc}
\chapter{Appendices}
\labch{appendix}


\section{Comprehensive table of Hoon runes}
\labsec{ap:runes}

These runes are up-to-date as of Hoon \texttt{\%140}.  Runes are ordered by alphabetical pronunciation of Urbit-standard aural ASCII.  Standard definitions (such as \texttt{+\$map} and \texttt{+\$unit}) are defined in Section~\ref{he:structures}.  Of those remaining, the most common is \texttt{+\$hoon}, which is used as a shorthand for any valid Hoon expression.  A \texttt{+\$term} is TODO tome spec value

Digraphs

\begin{table}[h]
  \caption{Aural ASCII Pronunciation}
  \label{}
  \begin{tabular}{clclclcl}
    \texttt{\textasciitilde} & “sig” &
    \texttt{!} & “zap” &
    \texttt{\@} & “pat” \\
  \end{tabular}
\end{table}

\subsection{\pbar:  Core Definition}
\labsec{ap:bar}

\pbar~runes produce cores.

\subsubsection{\barbuc\texttt{~(lest term) spec}} %%%%%%%%%%%%%%%%%%%%%%%%%%%%%%
\label{ap:barbuc}

\pbarbuc~produces a mold (a type definition) given a non-empty list or \texttt{+\$lest} of \texttt{+\$term}s (\pattas~ASCII symbols, the labels) and a structure definition or \texttt{+\$spec}.  In other words, \barbuc~

\paragraph{Definition}



\paragraph{Examples}

\begin{lstlisting}[caption={\texttt{++lest} (non-empty list) from \texttt{hoon.hoon}},
                   style=nonumbers]
++  lest
  |$  [item]
  ::    null-terminated non-empty list
  ::
  ::  mold generator: produces a mold of a null-terminated list of the
  ::  homogeneous type {a} with at least one element.
  [i=item t=(list item)]
\end{lstlisting}

\begin{lstlisting}[caption={\texttt{++pair} from \texttt{hoon.hoon}},
                   style=nonumbers]
++  pair
  |$  [head tail]
  ::    dual tuple
  ::
  ::  mold generator: produces a tuple of the two types passed in.
  ::
  ::  a: first type, labeled {p}
  ::  b: second type, labeled {q}
  ::
  [p=head q=tail]
\end{lstlisting}


\subsubsection{{\barcab\texttt{~spec alas (map term tome)}}} %%%%%%%%%%%%%%%%%%%
\label{ap:barcab}

\pbarcab produces a door (a core with sample) given a spec alas and a

\paragraph{Examples}

\begin{lstlisting}[style=nonumbers]
TODO
\end{lstlisting}

\begin{lstlisting}[caption={\texttt{+\$mk-item} from \texttt{hoon.hoon}},
                   style=nonumbers]
::  $mk-item: constructor for +ordered-map item type
++  mk-item
  |$  [key val]
  [key=key val=val]
\end{lstlisting}


\subsubsection{\textbf{\barcen\texttt{~(unit term) (map term tome)}}}
\label{ap:barcen}

\pbarcab~produces a core (battery and payload) given a \texttt{unit} of \texttt{term}s (\pattas~ASCII symbols, arm labels, the battery) and a dictionary of named arms, or \texttt{map} from \texttt{term} keys to \texttt{tome} values ().  In other words,


\subsubsection{\textbf{\barcol\texttt{~[p=hoon q=hoon]}}}
\label{ap:barcol}

\pbarcol~produces a gate with a custom sample given a

\paragraph{Definition}

\begin{lstlisting}
[%tsls p.gen [%brdt q.gen]]
=+  p  |. q
\end{lstlisting}

\subsubsection{\textbf{\bardot\texttt{~hoon}}}
\label{ap:bardot}
produces a trap (a core with one arm)

\subsubsection{\textbf{\barhep\texttt{~hoon}}}
\label{ap:barcen}
produces a trap (a core with one arm) and evaluates it

\subsubsection{\textbf{\barket\texttt{~hoon (map term tome)}}}
\label{ap:barcen}
produces a core whose battery includes a \$ arm and computes the latter

\subsubsection{\textbf{\barpat\texttt{~(unit term) (map term tome)}}}
\label{ap:barpat}
produces a wet core (battery and payload)

\subsubsection{\textbf{\barsig\texttt{~[spec value]}}}
\label{ap:barcen}
produces an iron gate

\subsubsection{\textbf{\bartar\texttt{~[spec value]}}}
\label{ap:barcen}
produces a wet gate (a one-armed core with sample)

\subsubsection{\textbf{\bartis\texttt{~[spec value]}}}
\label{ap:barcen}
produces a dry gate (a one-armed core with sample)

\subsubsection{\textbf{\barwut\texttt{~hoon}}}
\label{ap:barwut}

\pbarwut~produces a lead trap (bivariant) or core with a single arm \texttt{\$}

\lead~lets you export a library interface but hide the implementation details

%------------------------------------------------------------------------------%
\subsection{\pbuc:  Mold Definition}
\labsec{ap:buc}

\pbuc~runes produce mold definitions.

%------------------------------------------------------------------------------%
\subsection{\pcen:  Core Evaluation}
\labsec{ap:cen}

\pcen~runes evaluate cores (similar to function calls in other languages).

\begin{lstlisting}
(~(rad og eny) 5)
(rad:og 5)  :: TODO figure out tersest working equivalent here
\end{lstlisting}

%------------------------------------------------------------------------------%
\subsection{\pcol:  Cell Construction}
\labsec{ap:col}

\pcol~runes produce cells of various structures.

%------------------------------------------------------------------------------%
\subsection{\pdot:  Nock Evaluation}
\labsec{ap:dot}

\pdot~runes directly evaluate as Nock expressions.

%------------------------------------------------------------------------------%
\subsection{\pket:  Core Typecasting}
\labsec{ap:ket}

\pket~runes are used to alter cores.

%------------------------------------------------------------------------------%
\subsection{\psig:  Hinting}
\labsec{ap:sig}

\psig~runes represent directives to the runtime.

%------------------------------------------------------------------------------%
\subsection{\pmic:  Macro}
\labsec{ap:mic}

\mic~runes produce calling structures (mainly monadic binds) and XML elements.

%------------------------------------------------------------------------------%
\subsection{\ptis:  Composition}
\labsec{ap:tis}

\tis~runes modify the subject.

%------------------------------------------------------------------------------%
\subsection{\pwut:  Conditional}
\labsec{ap:wut}

\wut~runes compare subtrees.

%------------------------------------------------------------------------------%
\subsection{\pzap:  Wildcard}
\labsec{ap:zap}

\zap~runes perform a miscellaneous set of auxiliary operations.

%%%%%%%%%%%%%%%%%%%%%%%%%%%%%%%%%%%%%%%%%%%%%%%%%%%%%%%%%%%%%%%%%%%%%%%%%%%%%%%%
\section{Hoon versions}
\labsec{ap:hoonversions}

%------------------------------------------------------------------------------%

%%%%%%%%%%%%%%%%%%%%%%%%%%%%%%%%%%%%%%%%%%%%%%%%%%%%%%%%%%%%%%%%%%%%%%%%%%%%%%%%
\section{Nock versions}
\labsec{ap:nockversions}

%%%%%%%%%%%%%%%%%%%%%%%%%%%%%%%%%%%%%%%%%%%%%%%%%%%%%%%%%%%%%%%%%%%%%%%%%%%%%%%%
\section{Hoon comparison with other languages}
\labsec{ap:comparison}

%%%%%%%%%%%%%%%%%%%%%%%%%%%%%%%%%%%%%%%%%%%%%%%%%%%%%%%%%%%%%%%%%%%%%%%%%%%%%%%%
\section{\zuse/\lull~versions}
\labsec{ap:zuseversions}

%%%%%%%%%%%%%%%%%%%%%%%%%%%%%%%%%%%%%%%%%%%%%%%%%%%%%%%%%%%%%%%%%%%%%%%%%%%%%%%%
\section{Textbook changelog}
\labsec{ap:changelog}
