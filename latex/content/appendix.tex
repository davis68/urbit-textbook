\setchapterpreamble[u]{\margintoc}
\chapter{Appendices}
\labch{appendix}


\section{Comprehensive table of Hoon runes}
\labsec{ap:runes}

These runes are up-to-date as of Hoon \texttt{\%140}.  Runes are ordered by alphabetical pronunciation of Urbit-standard aural ASCII.  Standard definitions (such as \texttt{+\$map} and \texttt{+\$unit}) are defined in Section~\ref{he:structures}.  Of those remaining, the most common is \texttt{+\$hoon}, which is used as a shorthand for any valid Hoon expression.  A \texttt{+\$term} is TODO tome spec value

Digraphs

\begin{table}
  \caption{Aural ASCII Pronunciation}
  \label{}
  \begin{tabular}{clclclcl}
    \texttt{\textasciitilde} & “sig” &
    \texttt{!} & “zap” &
    \texttt{\@} & “pat” \\
  \end{tabular}
\end{table}

\subsection{\texttt{|} “bar”:  Core Definition}
\labsec{ap:bar}

\pbar~runes produce cores.

\begin{tabular}{l}
\hline \\ \hline \\
\textbf{\barbuc\texttt{~(lest term) spec}}
\\
\pbarbuc~produces a mold from a TODO.
\\
\begin{lstlisting}[style=nonumbers]
TODO
\end{lstlisting}
\\
\begin{lstlisting}[style=nonumbers]
::  $mk-item: constructor for +ordered-map item type
++  mk-item  |$  [key val]  [key=key val=val]
\end{lstlisting}
\\
\hhline{=} \\
\textbf{\barcab\texttt{~spec alas (map term tome)}}
\\ \hline
\\
\pbarcab produces a door (a core with sample) given XYZ
\\
\hline \\ \hline \\
\textbf{\barcen\texttt{~(unit term) (map term tome)}}
\\
produces a core (battery and payload)
\\
\hhline{=} \\
\textbf{\barcol\texttt{[hoon hoon]}}
\\
produces a gate with a custom sample
\hhline{=} \\
\textbf{\bardot\texttt{~hoon}}
\\
produces a trap (a core with one arm)
\hhline{=} \\
\textbf{\barhep\texttt{~hoon}}
\\
produces a trap (a core with one arm) and evaluates it
\hhline{=} \\
\textbf{\barket\texttt{~hoon (map term tome)}}
\\
produces a core whose battery includes a \$ arm and computes the latter
\hhline{=} \\
\textbf{\barpat\texttt{~(unit term) (map term tome)}}
\\
produces a wet core (battery and payload)
\hhline{=} \\
\textbf{\barsig\texttt{~[spec value]}}
\\
produces an iron gate
\hhline{=} \\
\textbf{\bartar\texttt{~[spec value]}}
\\
produces a wet gate (a one-armed core with sample)
\hhline{=} \\
\textbf{\bartis\texttt{~[spec value]}}
\\
produces a dry gate (a one-armed core with sample)
\\
\hhline{=} \\
\textbf{\barwut\texttt{~hoon}}
\\
produces a lead trap

\end{tabular}

%------------------------------------------------------------------------------%
\subsection{\texttt{\S} “buc”:  Mold Definition}
\labsec{ap:buc}

\pbuc~runes produce mold definitions.

%------------------------------------------------------------------------------%
\subsection{\texttt{\%} “cen”:  Core Evaluation}
\labsec{ap:cen}

\pcen~runes evaluate cores (similar to function calls in other languages).

\begin{lstlisting}
(~(rad og eny) 5)
(rad:og 5)  :: TODO figure out tersest working equivalent here
\end{lstlisting}

%------------------------------------------------------------------------------%
\subsection{\texttt{:} “col”:  Cell Construction}
\labsec{ap:col}

\pcol~runes produce cells of various structures.

%------------------------------------------------------------------------------%
\subsection{\texttt{.} “dot”:  Nock Evaluation}
\labsec{ap:dot}

\pdot~runes directly evaluate as Nock expressions.

%------------------------------------------------------------------------------%
\subsection{\texttt{\^} “ket”:  Core Typecasting}
\labsec{ap:ket}

\pket~runes are used to alter cores.

%------------------------------------------------------------------------------%
\subsection{\texttt{\textasciitilde} “sig”:  Hinting}
\labsec{ap:sig}

\psig~runes represent directives to the runtime.

%------------------------------------------------------------------------------%
\subsection{\texttt{;} “mic”:  Macro}
\labsec{ap:mic}

\mic~runes produce calling structures (mainly monadic binds) and XML elements.

%------------------------------------------------------------------------------%
\subsection{\texttt{=} “tis”:  Subject Alteration}
\labsec{ap:tis}

\tis~runes modify the subject.

%------------------------------------------------------------------------------%
\subsection{\texttt{?} “wut”:  Comparison}
\labsec{ap:wut}

\wut~runes compare subtrees.

%------------------------------------------------------------------------------%
\subsection{\texttt{!} “zap”:  Wildcard}
\labsec{ap:zap}

\zap~runes perform a miscellaneous set of auxiliary operations.

%%%%%%%%%%%%%%%%%%%%%%%%%%%%%%%%%%%%%%%%%%%%%%%%%%%%%%%%%%%%%%%%%%%%%%%%%%%%%%%%
\section{Hoon versions}
\labsec{ap:hoonversions}

%------------------------------------------------------------------------------%

%%%%%%%%%%%%%%%%%%%%%%%%%%%%%%%%%%%%%%%%%%%%%%%%%%%%%%%%%%%%%%%%%%%%%%%%%%%%%%%%
\section{Nock versions}
\labsec{ap:nockversions}

%%%%%%%%%%%%%%%%%%%%%%%%%%%%%%%%%%%%%%%%%%%%%%%%%%%%%%%%%%%%%%%%%%%%%%%%%%%%%%%%
\section{Hoon comparison with other languages}
\labsec{ap:comparison}

%%%%%%%%%%%%%%%%%%%%%%%%%%%%%%%%%%%%%%%%%%%%%%%%%%%%%%%%%%%%%%%%%%%%%%%%%%%%%%%%
\section{\zuse/\lull~versions}
\labsec{ap:zuseversions}

%%%%%%%%%%%%%%%%%%%%%%%%%%%%%%%%%%%%%%%%%%%%%%%%%%%%%%%%%%%%%%%%%%%%%%%%%%%%%%%%
\section{Textbook changelog}
\labsec{ap:changelog}
