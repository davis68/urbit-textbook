\setchapterpreamble[u]{\margintoc}
\chapter{Userspace}
\labch{Userspace}

\section{\gall, A Runtime Agent}
\labsec{us:gall}

Userspace applications are conceptually (but not architecturally) divisible into three categories.  It is recommended as a burgeoning best practice that new apps hew to this division for clarity and forward compatibility.

\begin{enumerate}
  \item  \textbf{Stores}.  Independent local collections of data exposed to other apps.  Think of Groups for Landscape.
  \item  \textbf{Hooks}.  Store manipulations.  Think of subscriptions.
  \item  \textbf{Views}.  User-interface applications.  Think of Landscape itself (the browser interface).
\end{enumerate}

\href{https://docs.google.com/document/d/1hS_UuResG1S4j49_H-aSshoTOROKBnGoJAaRgOipf54/edit}{Logan Allen, Matt, Matilde Park `~haddef-sigwen`, "Userspace Architecture"}

\marginnote[2mm]{
Modern \gall~is sometimes called "static Gall," in contrast to an earlier specification "dynamic Gall."  Dynamic Gall did not specify the arms and permitted each agent its own structure; in practice, this proved to be difficult for programmers to maintain in a consistent manner, leading to code refactors and maintenance of defunct arms for backwards compatibility of agents.
}

All userspace apps are mediated by \gall, which provides a standard set of operations to interface with userspace applications.  \gall's domain essentially consists of user-visible state machines.  Almost everything you interact with as a user is mediated through \gall:  Chat, Publish, Dojo, and so forth.

% TODO diagrams etc.

Every complex structure in Hoon is a core; a \gall~agent structurally requires ten arms in its main core with an optional helper core.  The agent itself is a door with two components in its subject:

\begin{enumerate}
  \item  \texttt{bowl:gall} defined \gall-standard tools and data structures.
  \item  Agent state information (with version number)
\end{enumerate}

A \gall~agent can use default arms provided by the \texttt{default-agent} library in case they are not needed.  For instance, a minimalist \gall~agent looks like this:

\begin{lstlisting}
/+  default-agent
^-  agent:gall
=|  state=@  :: TODO still valid?
|_  =bowl:gall
+*  this      .
    default   ~(. (default-agent this %|) bowl)
::
++  on-init   on-init:default
++  on-save   on-save:default
++  on-load   on-load:default
++  on-poke
  |=  [=mark =vase]
  ~&  >  state=state
  ~&  got-poked-with-data=mark
  =.  state  +(state)
  `this
::
++  on-watch  on-watch:default
++  on-leave  on-leave:default
++  on-peek   on-peek:default
++  on-agent  on-agent:default
++  on-arvo   on-arvo:default
++  on-fail   on-fail:default
\end{lstlisting}


\marginnote[2mm]{
Compare this to other core structures:  a gate is `[battery [sample context]]`; a move is `[bone [tag noun]]`.
}

For \gall, a standard move is a pair \texttt{[bone card]}.  A \texttt{bone} is a cause while a \texttt{card} is an action.  (\texttt{[cause action]}, it's always \texttt{[cause action]}.)  A \texttt{card}, in turn, is a pair \texttt{[tag noun]}.  The tag is a \pattas token to indicate which event is being triggered while the \texttt{noun} contains any necessary information.

\gall~apps run like services, so they can always talk to each other.  This leads to a very high potential for app interactivity rather than siloization.

\subsection{Scrying into \gall}
\labsec{us:g:scry}


\gall~possesses several scry types, although not as rich as \clay:

\begin{table}
  \begin{center}
    \caption{\gall~\dotket~Calls.}
    \label{ha:gall}
    \begin{tabular}{lll}
      Symbol & Meaning & Example \\
      \hline \\
      \texttt{\%e} &  & \\
      \texttt{\%x} & ; returns a \texttt{vase} compatible with the mark at the end of the scry request. & \\
      \texttt{\%y} & ; returns a \texttt{cage} with a mark of \texttt{\%arch} and a \texttt{vase} of \texttt{\%arch}. & \\
    \end{tabular}
  \end{center}
\end{table}


\subsection{Structures, Patterns, and Factories}
\labsec{us:patterns}

\emph{Structures} define common patterns of interaction which are shared across system components and agents.  A structure file lives in \texttt{/sur}, and is generally distinguished from being a \texttt{/lib} library file in that it only presents molds as type definitions, whereas a library contains operable code as gates and doors and the like.  (This is by convention not force.)  Structures are imported using the \pfaslus~rune.  For instance, consider the \texttt{/sur/file-server/hoon}, which defines \texttt{action}s, \texttt{configuration}s, and \texttt{update}s for the \texttt{\%file-server} \gall~app.

\begin{lstlisting}[caption={\texttt{/sur/file-server/hoon}}]
/-  glob
|%
+$  action
  $%  [%serve-dir url-base=path clay-base=path public=? spa=?]
      [%serve-glob url-base=path =glob:glob public=?]
      [%unserve-dir url-base=path]
      [%toggle-permission url-base=path]
      [%set-landscape-homepage-prefix prefix=(unit term)]
  ==
::
+$  configuration
  $:  landscape-homepage-prefix=(unit term)
  ==
::
+$  update
  $%  [%configuration =configuration]
  ==
--
\end{lstlisting}

This in turn relies on the \texttt{/sur/glob/hoon} structure defining a \texttt{glob}.  (In Unix parlance, a \emph{glob} is a collection of files, such as that obtained by matching a regular expression, such as \texttt{*.txt}.)

\begin{lstlisting}[caption={\texttt{/sur/glob/hoon}}]
|%
+$  glob  (map path mime)
--
\end{lstlisting}



\emph{Factories} are code patterns that are commonly employed in creating
In object-oriented programming, these are object-constructing objects.  In subject-oriented programming, one uses wet cores to create the appropriate newly patterned core.

One simple constructed example (that is, not a true factory) is the use of \texttt{default} core to produce \gall~agents.

TODO

\section{Deep Dives in \gall}

Several of the following case studies is drawn from published code, most of it incorporated into the Urbit userspace.  In some cases, the original code uses conventions we have not yet introduced; we have simplified these to rely on the runes introduced in the main text through Chapter~\ref{ha}.  Other examples are new to this chapter.

\subsection{\shoe/\sole~CLI Libraries}
\labsec{us:shoe}

\marginnote[2mm]{
See also the \url{https://urbit.org/docs/hoon/guides/cli-tutorial/}{Tlon \shoe~tutorial} for an alternative explanation of these principles.
}
Prefatory to examining the mechanics of \texttt{\%chat-cli} and the Dojo terminal parser, we need to consider what a command-line interface is and what functionality it offers.  We will then construct a simple reverse-Polish notation (RPN) calculator as a minimal working example of a \shoe-instrumented CLI app.


\subsubsection{Reverse-Polish Notation Calculator, \texttt{\%pyth}}

We wish to create a calculator which accepts one or two numbers as input, then an operator which evaluates the one or two preceding numbers on the stack, as appropriate.  (For instance, unary negation could be input as \texttt{5 -}.)  We will implement the following operators:

\begin{enumerate}
  \item  \lus, addition (binary operator)
  \item  \hep, subtraction (unary if only one number, else binary)
  \item  \tar, multiplication (binary operator)
  \item  \fas, division (binary operator)
  \item  \ket, exponentiation (binary operator)
  \item  \cab, square root (unary operator)
\end{enumerate}

Like Dojo, we do not need to accept any invalid input, but we do need to deal with some special cases:  automatically promoting integer input values (\patud) to double-precision floating-point numbers (\patrd), and deciding whether \texttt{-} refers to a unary or binary operation.  Input of a number should pause until completion is indicated by pressing Return, but an operator may be evaluated instantly.

For instance, here is an expected session, with \texttt{\%} being the CLI prompt:

\begin{lstlisting}[style=nonumbers]
% 5
5.0
% 4
5.0 4.0
% -
1.0
% _
1.0
% -
-1.0
% 6
-1.0 6.0
% 12
-1.0 6.0 12.0
% *
-1.0 72.0
% /
-0.013888888888888888
\end{lstlisting}

We construct \texttt{\%pyth} beginning with the default \gall~core.  Save this text as a file \texttt{app/pyth.hoon}.

\begin{lstlisting}[caption={Default \gall~core for \texttt{\%pyth}}]
/+  default-agent
^-  agent:gall
=|  state=@  :: TODO still valid?
|_  =bowl:gall
+*  this      .
    default   ~(. (default-agent this %|) bowl)
::
++  on-init   on-init:default
++  on-save   on-save:default
++  on-load   on-load:default
++  on-poke
  |=  [=mark =vase]
  ~&  >  state=state
  ~&  got-poked-with-data=mark
  =.  state  +(state)
  `this
::
++  on-watch  on-watch:default
++  on-leave  on-leave:default
++  on-peek   on-peek:default
++  on-agent  on-agent:default
++  on-arvo   on-arvo:default
++  on-fail   on-fail:default
\end{lstlisting}

\texttt{\%pyth} needs to set the following arms to play their roles:

\begin{enumerate}
  \item  \texttt{++on-init} should initialize an empty stack.
  \item  \texttt{++on-load} should clear the stack.
  \item  \texttt{++on-save} should push the current state for an upgrade.
  \item  \texttt{++on-poke}
  \item  \texttt{++on-agent}
\end{enumerate}

Other arms may remain with their default settings.

\texttt{++on-init} is responsible to set up the prompt symbol \texttt{\%} and define the parsing rules.

\texttt{++on-poke} is the workhorse which takes any input and decides how to parse it, as well as either pushing numeric values onto the stack or effecting operators.


\subsection{Chat CLI}
\labsec{us:chat}

\subsection{Drum, Helm, Hood, and Herb}
\labsec{us:drum}

Collectively, these agents are components of or adjacent to Dojo which permit text input (from the keyboard via \dill~or through a plaintext API).

\subsection{Bitcoin API}
\labsec{us:btc}

\subsection{Ranked Voting}
\labsec{us:voting}

Ranked voting describes a class of voting systems wherein participants can designate primary, secondary, and perhaps more votes in an election, ranked by preference.  Various schemes consolidate these votes into a final assessment of the winning item.  In this example, we will create a \gall~agent to implement a \emph{positional voting} algorithm.  We permit voters to rank three choices of many, with the first choice receiving 3 points, the second receiving 2 points, and the third choice receiving 1 point.  Other choices receive no points.  (This variant is called a Borda count.)  There are two roles in the system:  a \emph{sponsor} \texttt{\%sponsor} who proposes a ballot, including populating the selection candidates and finalizing the vote; and one or more \emph{voters} \texttt{\%voter}, possibly including the sponsor, who vote on the items.

\begin{example}

The \gall~agent

Furthermore, the agent needs to interact with an external browser-based interface or CLI.  In this case, we elect to compose a CLI.

\lstset{title=\texttt{ranked-choice}}
\begin{lstlisting}
/+  default-agent, dbug
|%
+$  versioned-state
    $%  state-0
    ==
+$  state-0  [%0 counter=@]
--
%-  agent:dbug
=|  state-0
=*  state  -
^-  agent:gall
|_  =bowl:gall
+*  this      .
    default   ~(. (default-agent this %|) bowl)
::
++  on-init   >  'on-init'
  `this(state [%0 3])
++  on-save  ^-  vase  !>(state)
++  on-load
  ~&  >  'on-load'
  on-load:default
++  on-poke
  |=  [=mark =vase]
  ~&  >  state=state
  ~&  got-poked-with-data=mark
  =.  state  +(state)
  `this
  :: =/  vote
  :: ?:  (in '')
  ::
::
++  on-watch  on-watch:default
++  on-leave  on-leave:default
++  on-peek   on-peek:default
++  on-agent  on-agent:default
++  on-arvo   on-arvo:default
++  on-fail   on-fail:default
\end{lstlisting}

\end{example}


\subsection{Bots}
\labsec{us:bots}

\section{Threading with Spider}
\labsec{us:spider}

\section{Urbit API}
\labsec{us:api}

\section{Deep Dives with Urbit API}
\labsec{us:api-deep}

\subsection{Time (Clock)}
\labsec{us:time}

\subsection{Publish}
\labsec{us:publish}

\subsection{\graphstore}
\labsec{us:graph}

Logan:  they’re all enumerated in the +on-peek arm of graph-store




\subsection{WebRTC Applications}
\labsec{us:webrtc}

Initialize Urbit airlock http-api
Create UrbitRTCApp instance
Register handler for incomingcall event
  reject/answer
Call initialize()
Place calls using call()
or Start with peer connection from answer()
Create Icepond instance
Register handler for iceserver events
Add media or data streams
Call initialize on Icepond and peer connection

commoditizes external infrastructure dependencies
unifies personal identity and address
removes centralized coordination for call negotiation (peer-to-peer media calls)
(bootstrapping from Urbit peer-to-peer cnxn)


\section{Exercises}
\labsec{us:exercises}

\begin{enumerate}
  \item  Produce a ranked-choice voting agent which implements instant-runoff voting.  In this version, voters rank all items.  A majority item wins outright ($>50\%$); otherwise all votes for the item with the fewest first choices are redistributed based on which item is next on each voter's ballot.  This proceeds until a winner is determined.
  \item  Produce a command-line calculator which may be accessed by anyone.
\end{enumerate}
